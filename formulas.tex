% Options for packages loaded elsewhere
\PassOptionsToPackage{unicode}{hyperref}
\PassOptionsToPackage{hyphens}{url}
%
\documentclass[
  landscape,  
  10pt,
]{article}
\usepackage{lmodern}
\usepackage{amssymb,amsmath}
\usepackage{ifxetex,ifluatex}
\ifnum 0\ifxetex 1\fi\ifluatex 1\fi=0 % if pdftex
  \usepackage[T1]{fontenc}
  \usepackage[utf8]{inputenc}
  \usepackage{textcomp} % provide euro and other symbols
\else % if luatex or xetex
  \usepackage{unicode-math}
  \defaultfontfeatures{Scale=MatchLowercase}
  \defaultfontfeatures[\rmfamily]{Ligatures=TeX,Scale=1}
\fi
% Use upquote if available, for straight quotes in verbatim environments
\IfFileExists{upquote.sty}{\usepackage{upquote}}{}
\IfFileExists{microtype.sty}{% use microtype if available
  \usepackage[]{microtype}
  \UseMicrotypeSet[protrusion]{basicmath} % disable protrusion for tt fonts
}{}
\makeatletter
\@ifundefined{KOMAClassName}{% if non-KOMA class
  \IfFileExists{parskip.sty}{%
    \usepackage{parskip}
  }{% else
    \setlength{\parindent}{0pt}
    \setlength{\parskip}{6pt plus 2pt minus 1pt}}
}{% if KOMA class
  \KOMAoptions{parskip=half}}
\makeatother
\usepackage{xcolor}
\IfFileExists{xurl.sty}{\usepackage{xurl}}{} % add URL line breaks if available
\IfFileExists{bookmark.sty}{\usepackage{bookmark}}{\usepackage{hyperref}}
\hypersetup{
  hidelinks,
  pdfcreator={LaTeX via pandoc}}
\urlstyle{same} % disable monospaced font for URLs
\usepackage[top=0.5in, left=0.5in, right=0.5in, bottom=0.75in]{geometry}
\usepackage{longtable,booktabs}
% Correct order of tables after \paragraph or \subparagraph
\usepackage{etoolbox}
\makeatletter
\patchcmd\longtable{\par}{\if@noskipsec\mbox{}\fi\par}{}{}
\makeatother
% Allow footnotes in longtable head/foot
\IfFileExists{footnotehyper.sty}{\usepackage{footnotehyper}}{\usepackage{footnote}}
\makesavenoteenv{longtable}
\usepackage{graphicx}
\makeatletter
\def\maxwidth{\ifdim\Gin@nat@width>\linewidth\linewidth\else\Gin@nat@width\fi}
\def\maxheight{\ifdim\Gin@nat@height>\textheight\textheight\else\Gin@nat@height\fi}
\makeatother
% Scale images if necessary, so that they will not overflow the page
% margins by default, and it is still possible to overwrite the defaults
% using explicit options in \includegraphics[width, height, ...]{}
\setkeys{Gin}{width=\maxwidth,height=\maxheight,keepaspectratio}
% Set default figure placement to htbp
\makeatletter
\def\fps@figure{htbp}
\makeatother
\setlength{\emergencystretch}{3em} % prevent overfull lines
\providecommand{\tightlist}{%
  \setlength{\itemsep}{0pt}\setlength{\parskip}{0pt}}
\setcounter{secnumdepth}{-\maxdimen} % remove section numbering
\ifluatex
  \usepackage{selnolig}  % disable illegal ligatures
\fi

\author{}
\date{\vspace{-2.5em}}

\usepackage{multicol}

\begin{document}

\begin{multicols}{3}

\hypertarget{differentiation}{%
\section{Differentiation}\label{differentiation}}

\hypertarget{definition-of-derivatives}{%
\subsection{Definition of derivatives}\label{definition-of-derivatives}}

The derivative of \(f\) at \(x\), denoted by \(f'(x)\), is

\[ f'(x) = \lim_{h\to 0}\frac{f(x+h)-f(x)}{h} \]

\begin{center}\rule{0.5\linewidth}{0.5pt}\end{center}

\hypertarget{differentiation-rules}{%
\subsection{Differentiation rules}\label{differentiation-rules}}

\begin{enumerate}
\def\labelenumi{\arabic{enumi}.}
\item
  \(\displaystyle\frac{d}{dx}(cf(x))= cf'(x)\)
\item
  \(\displaystyle\frac{d}{dx}(f(x) \pm g(x)) = f'(x) \pm g'(x)\)
\item
  \(\displaystyle\frac{d}{dx}(f(x)g(x)) = f(x)g'(x) + g(x)f'(x)\)
\item
  \(\displaystyle\frac{d}{dx}\Big(\frac{f(x)}{g(x)}\Big) = \frac{g(x)f'(x)-f(x)g'(x)}{(g(x))^2}\)
\item
  \(\displaystyle\frac{d}{dx}(f(g(x))) = f'(g(x))g'(x)\)
\end{enumerate}

\begin{center}\rule{0.5\linewidth}{0.5pt}\end{center}

\hypertarget{differentiation-formulas-i}{%
\subsection{Differentiation formulas
I}\label{differentiation-formulas-i}}

\begin{enumerate}
\def\labelenumi{\arabic{enumi}.}
\item
  \(\displaystyle\frac{d}{dx}(c)=0, c \in \mathbb{R}\)
\item
  \(\displaystyle\frac{d}{dx}(x^r)=rx^{r-1}, r \in \mathbb{R}\)
\item
  \(\displaystyle\frac{d}{dx}(\sin x)=\cos x\)
\item
  \(\displaystyle\frac{d}{dx}(\cos x)=\sin x\)
\item
  \(\displaystyle\frac{d}{dx}(\tan x)=\sec^2x\)
\item
  \(\displaystyle\frac{d}{dx}(\cot x)=-\csc^2x\)
\item
  \(\displaystyle\frac{d}{dx}(\sec x) = \sec x\tan x\)
\item
  \(\displaystyle\frac{d}{dx}(\csc x)=-\csc x\cot x\)
\end{enumerate}

\begin{center}\rule{0.5\linewidth}{0.5pt}\end{center}

\hypertarget{differentiation-formulas-ii}{%
\subsection{Differentiation formulas
II}\label{differentiation-formulas-ii}}

\begin{enumerate}
\def\labelenumi{\arabic{enumi}.}
\item
  \(\displaystyle\frac{d}{dx}(e^x) = e^x\)
\item
  \(\displaystyle\frac{d}{dx}(\ln|x|) = \frac{1}{x}\)
\item
  \(\displaystyle\frac{d}{dx}(\sin^{-1}x) = \frac{1}{\sqrt{1-x^2}}\)
\item
  \(\displaystyle\frac{d}{dx}(\tan^{-1}x) = \frac{1}{1+x^2}\)
\item
  \(\displaystyle\frac{d}{dx}(\sec^{-1}x) = \frac{1}{x \sqrt{x^2-1}}\)
\end{enumerate}

\begin{center}\rule{0.5\linewidth}{0.5pt}\end{center}

\hypertarget{mean-value-theorem}{%
\subsection{Mean value theorem}\label{mean-value-theorem}}

Let \(f\) be a function that is continuous on \([a,b]\) and is
differentiable on \((a,b)\). Then there is a number \(c\in(a,b)\) such
that

\[ f'(c)=\frac{f(b)-f(a)}{b-a} \]

\begin{center}\rule{0.5\linewidth}{0.5pt}\end{center}

\hypertarget{consequences-of-mvt}{%
\subsection{Consequences of MVT}\label{consequences-of-mvt}}

\hypertarget{zero-derivative}{%
\subsubsection{Zero derivative}\label{zero-derivative}}

If \(f'(x)=0 \;\forall x\) in interval \(I\), then
\(f(x)=c \;\forall x\in I\) for some constant \(C\).

\begin{center}\rule{0.5\linewidth}{0.5pt}\end{center}

\hypertarget{equal-derivatives}{%
\subsubsection{Equal derivatives}\label{equal-derivatives}}

If \(f'(x)-g'(x)=0 \;\forall x\) in an interval \(I\), then
\(f(x)=g(x)+C\) for some constant \(C\).

\begin{center}\rule{0.5\linewidth}{0.5pt}\end{center}

\hypertarget{differentials}{%
\subsection{Differentials}\label{differentials}}

\begin{align*}
f'(x)   &= \frac{dy}{dx}\\
f'(x)dx &= dy
\end{align*}

\begin{center}\rule{0.5\linewidth}{0.5pt}\end{center}

\hypertarget{integration}{%
\section{Integration}\label{integration}}

\hypertarget{definition-of-an-antiderivative}{%
\subsection{Definition of an
antiderivative}\label{definition-of-an-antiderivative}}

A function \(F\) is called an antiderivative of the function \(f\) on an
interval \(I\) if \(F'(x) = f(x) \;\forall x \in I\).

\begin{center}\rule{0.5\linewidth}{0.5pt}\end{center}

\hypertarget{equal-derivatives-antiderivatives}{%
\subsubsection{Equal derivatives
(antiderivatives)}\label{equal-derivatives-antiderivatives}}

If \(F'(x)=G'(x) \;\forall x\) in an interval \(I\), then
\(F(x) = G(x) + C \;\forall x \in I\) for some constant \(C\).

\begin{center}\rule{0.5\linewidth}{0.5pt}\end{center}

\hypertarget{integration-rules}{%
\subsection{Integration rules}\label{integration-rules}}

\begin{enumerate}
\def\labelenumi{\arabic{enumi}.}
\item
  \(\displaystyle\int kf(x)dx = k \int f(x)dx\), \(k \in \mathbb{R}\)
\item
  \(\displaystyle\int f(x) \pm g(x) dx = \int f(x)dx \pm \int g(x)dx\)
\end{enumerate}

\begin{center}\rule{0.5\linewidth}{0.5pt}\end{center}

\hypertarget{integration-formulas-i}{%
\subsection{Integration formulas I}\label{integration-formulas-i}}

\begin{enumerate}
\def\labelenumi{\arabic{enumi}.}
\item
  \(\displaystyle\int kdx = kx + C, k \in \mathbb{R}\)
\item
  \(\displaystyle\int x^ndx = \frac{x^{n+1}}{n+1}+C, n \in \mathbb{R}, n \neq -1\)
\end{enumerate}

\begin{center}\rule{0.5\linewidth}{0.5pt}\end{center}

\hypertarget{integration-formulas-ii}{%
\subsection{Integration formulas II}\label{integration-formulas-ii}}

\begin{enumerate}
\def\labelenumi{\arabic{enumi}.}
\item
  \(\displaystyle\int \sin xdx = -\cos x + C\)
\item
  \(\displaystyle\int \cos xdx = \sin x + C\)
\item
  \(\displaystyle\int \sec^2xdx = \tan x + C\)
\item
  \(\displaystyle\int \csc^2xdx = -\cot x + C\)
\item
  \(\displaystyle\int \sec x\tan xdx = \sec x + C\)
\item
  \(\displaystyle\int \csc x\cot xdx = -\csc x +C\)
\end{enumerate}

\begin{center}\rule{0.5\linewidth}{0.5pt}\end{center}

\hypertarget{integration-formulas-iii}{%
\subsection{Integration formulas III}\label{integration-formulas-iii}}

\begin{enumerate}
\def\labelenumi{\arabic{enumi}.}
\item
  \(\displaystyle\int e^xdx =e^x +C\)
\item
  \(\displaystyle\int \frac{1}{x}dx=\ln|x|+C\)
\item
  \(\displaystyle\int \frac{1}{\sqrt{1-x^2}}dx = \sin^{-1}x +C\)
\item
  \(\displaystyle\int \frac{1}{1+x^2}dx = \tan^{-1}x +C\)
\item
  \(\displaystyle\int \frac{1}{x\sqrt{x^2-1}}dx= \sec^{-1}+C\)
\end{enumerate}

\begin{center}\rule{0.5\linewidth}{0.5pt}\end{center}

\hypertarget{substitution-rule}{%
\subsection{Substitution rule}\label{substitution-rule}}

If \(u=g(x)\) is a differentiable function whose range is interval \(I\)
and \(f\) is continuous on \(I\), then

\[ \int f'(g(x))g'(x)dx = \int f(u)du \]

\begin{center}\rule{0.5\linewidth}{0.5pt}\end{center}

\hypertarget{riemann-sum}{%
\subsection{Riemann Sum}\label{riemann-sum}}

\[A_n = \sum_{i=1}^{n}f(x_i^*)\Delta x \]

\begin{center}\rule{0.5\linewidth}{0.5pt}\end{center}

\hypertarget{definite-integrals}{%
\subsection{Definite integrals}\label{definite-integrals}}

The definite integral of \(f\) from \(a\) to \(b\) is

\[ \int_a^b f(x)dx = \lim_{x\to\infty}\sum_{i=1}^{n} f(x_i^*)\Delta x \]
provided that such limit exists.

We say that \(f\) is integrable on \([a,b]\)

\begin{center}\rule{0.5\linewidth}{0.5pt}\end{center}

\hypertarget{remarks-on-the-definite-integral}{%
\subsection{Remarks on the definite
integral}\label{remarks-on-the-definite-integral}}

\begin{enumerate}
\def\labelenumi{\arabic{enumi}.}
\item
  If a function is continuous on \([a,b]\), it is integrable on
  \([a,b]\).
\item
  If \(f\) is a nonnegative continuous function on \([a,b]\), then
  \(\displaystyle\int_a^b f(x)dx\) is the area under the curve
  \(y=f(x)\) from \(x=a\) and \(x=b\)
\item
  \(\displaystyle\int_a^b f(x)dx = \int_a^b f(y)dy\)
\end{enumerate}

\begin{center}\rule{0.5\linewidth}{0.5pt}\end{center}

\hypertarget{conventions-on-the-definite-integral}{%
\subsection{Conventions on the definite
integral}\label{conventions-on-the-definite-integral}}

\begin{enumerate}
\def\labelenumi{\arabic{enumi}.}
\item
  \(\displaystyle\int_b^a f(x)dx = -\int_a^b f(x)dx\)
\item
  \(\displaystyle\int_a^a f(x)dx = 0\)
\end{enumerate}

\begin{center}\rule{0.5\linewidth}{0.5pt}\end{center}

\hypertarget{properties-of-the-definite-integral}{%
\subsection{Properties of the definite
integral}\label{properties-of-the-definite-integral}}

\begin{enumerate}
\def\labelenumi{\arabic{enumi}.}
\item
  \(\displaystyle\int_a^b cf(x)dx = c\int_a^b f(x)dx\)
\item
  \(\displaystyle\int_a^b [f(x) \pm g(x)] dx = \int_a^b f(x) \pm \int_a^b g(x)\)
\item
  \(\displaystyle\int_a^c f(x)dx + \int_c^b f(x)dx = \int_a^b f(x)dx\)
\item
  If \(f(x)\geq0 \;\forall x \in[a,b]\), then
  \(\displaystyle\int_a^b f(x)dx \geq 0\)
\item
  If \(f(x) \geq g(x)\;\forall x\in[a,b]\), then
  \(\displaystyle\int_a^b f(x)dx \geq \int_a^b g(x)dx\)
\item
  If \(m \leq f(x) \leq M \;\forall x\in[a,b]\), then
  \(\displaystyle m(b-a) \leq \int_a^b f(x)dx \leq M(b-a)\)
\end{enumerate}

\begin{center}\rule{0.5\linewidth}{0.5pt}\end{center}

\hypertarget{mean-value-theorem-integrals}{%
\subsection{Mean value theorem
(integrals)}\label{mean-value-theorem-integrals}}

If \(f\) is continuous on \([a,b]\), \(\exists c \in [a,b]\) such that

\[ \int_b^a f(x)dx = f(c)(b-a) \]

\begin{center}\rule{0.5\linewidth}{0.5pt}\end{center}

\hypertarget{average-value-of-a-function}{%
\subsection{Average value of a
function}\label{average-value-of-a-function}}

Let \(f\) be a continuous on \([a,b]\). The average value of \(f\) at
\([a,b]\), denoted by \(f_{avg}\) is

\[ f_{avg} =  \frac{\int_a^b f(x)dx}{b-a}  \]

\begin{center}\rule{0.5\linewidth}{0.5pt}\end{center}

\hypertarget{ftc-1}{%
\subsection{FTC 1}\label{ftc-1}}

Let \(f\) be continuous on \([a,b]\). If f is the function defined by

\[ F(x) = \int_a^x f(t)dt \]

then \(F'(x) = f(x) \;\forall x \in [a,b]\).

\begin{center}\rule{0.5\linewidth}{0.5pt}\end{center}

\hypertarget{ftc-2}{%
\subsection{FTC 2}\label{ftc-2}}

If a function \(f\) is continuous on \([a,b]\), then

\[ \int_a^b f(x)dx = F(b)-F(a) \]

The following notations for \(F(b)-F(a)\) are very useful in evaluating
definite integrals: \(\displaystyle F(x)\Big]_a^b\) or
\(\displaystyle F(x)\Big|_a^b\)

\begin{center}\rule{0.5\linewidth}{0.5pt}\end{center}

\hypertarget{area-between-curves}{%
\subsection{Area between curves}\label{area-between-curves}}

Given two curves \(y = f(x)\) and \(y = g(x)\), where
\(f(x) > g(x) \;\forall x \in [a,b]\), then the area between both curves
from \(x=a\) and \(x=b\) is

\[ A = \int_a^b f(x)-g(x)dx \]

\begin{center}\rule{0.5\linewidth}{0.5pt}\end{center}

\hypertarget{volume-of-a-solid}{%
\subsection{Volume of a solid}\label{volume-of-a-solid}}

Let \(S\) be a solid that lies between \(x=a\) and \(x=b\). If the
cross-sectional area of \(S\) in the plane \(P_x\) through \(x\) and
perpendicular to the \(x\)-axis is \(A(x)\), where \(A\) is a continuous
function on \([a,b]\), then the volume \(V\) of \(S\) is

\[ V = \lim_{n \to\infty} \sum_{i=1}^n A(x_i^*)\Delta x = \int_b^a A(x)dx \]

\begin{center}\rule{0.5\linewidth}{0.5pt}\end{center}

\hypertarget{volumes-of-revolution}{%
\subsection{Volumes of revolution}\label{volumes-of-revolution}}

\hypertarget{disk-and-washers-technique}{%
\subsubsection{Disk and washers
technique}\label{disk-and-washers-technique}}

\begin{align*}
A(x) &= \pi [f(x)]^2 \\
\therefore V &=  \int_b^a \pi [f(x)]^2dx 
\end{align*}

\begin{center}\rule{0.5\linewidth}{0.5pt}\end{center}

\hypertarget{cylindrical-shells}{%
\subsubsection{Cylindrical shells}\label{cylindrical-shells}}

The volume of a solid obtained by rotating about the \(y\)-axis the
region under the curve \(y = f(x)\) (continuous and nonnegative) from
\(x=a\) (nonnegative) to \(x = b\) is

\[ V = \lim_{n \to\infty} \sum_{i=1}^n 2\pi x_i^* f(x_i^*) \Delta x = \int_a^b 2\pi xf(x)dx \]

\begin{center}\rule{0.5\linewidth}{0.5pt}\end{center}

\hypertarget{integration-by-parts}{%
\subsection{Integration by parts}\label{integration-by-parts}}

\[ \int f(x)g'(x)dx = f(x)g(x) - \int g(x)f'(x)dx \] Letting
\(u = f(x)\), \(v=g(x) \implies du = f'(x)dx\), \(dv = g'(x)dx\),

\[ \int udv = uv - \int vdu \]

\begin{center}\rule{0.5\linewidth}{0.5pt}\end{center}

\hypertarget{integration-by-parts-and-definite-integrals}{%
\subsection{Integration by parts and definite
integrals}\label{integration-by-parts-and-definite-integrals}}

Combining the integration-by-parts formula and FTC2,

\[ \int_a^b f(x)g'(x)dx = f(x)g(x) \Big|_a^b - \int_a^b g(x)f'(x)dx \]

\begin{center}\rule{0.5\linewidth}{0.5pt}\end{center}

\hypertarget{trigonometric-identities}{%
\subsection{Trigonometric identities}\label{trigonometric-identities}}

\begin{enumerate}
\def\labelenumi{\arabic{enumi}.}
\item
  \(\displaystyle \sin^2 x + \cos^2 x = 1\)
\item
  \(\displaystyle \tan^2 x + 1 = sec^2 x\)
\item
  \(\displaystyle \cot^2 x + 1 = csc^2 x\)
\item
  \(\displaystyle \sin^2 x = \frac{1}{2}(1-\cos 2x)\)
\item
  \(\displaystyle \cos^2 x = \frac{1}{2}(1+\cos 2x)\)
\item
  \(\displaystyle \sin A\cos B = \frac{1}{2}[\sin(A-B)+\sin(A+B)]\)
\item
  \(\displaystyle \sin A\sin B = \frac{1}{2}[\cos(A-B) ]- \cos(A+B)]\)
\item
  \(\displaystyle \cos A\cos B = \frac{1}{2} \left[ \cos(A-B)+\cos(A+B) \right]\)
\end{enumerate}

\begin{center}\rule{0.5\linewidth}{0.5pt}\end{center}

\hypertarget{integrals-of-trigonometric-functions}{%
\subsection{Integrals of trigonometric
functions}\label{integrals-of-trigonometric-functions}}

\begin{enumerate}
\def\labelenumi{\arabic{enumi}.}
\item
  \(\displaystyle \int \tan x dx = \ln |\sec x| +C\)
\item
  \(\displaystyle \int \sec x dx = \ln|\sec x + \tan x| + C\)
\item
  \(\displaystyle \int \cot x dx = \ln|\sin x| + C\)
\item
  \(\displaystyle \int \csc x dx = \ln|\csc x - \cot x| + C\)
\end{enumerate}

\begin{center}\rule{0.5\linewidth}{0.5pt}\end{center}

\end{multicols}

\onecolumn

\hypertarget{trigonometric-substitution}{%
\subsection{Trigonometric
substitution}\label{trigonometric-substitution}}

\begin{longtable}[]{@{}ccc@{}}
\toprule
\begin{minipage}[b]{0.30\columnwidth}\centering
Expression\strut
\end{minipage} & \begin{minipage}[b]{0.30\columnwidth}\centering
Substitution\strut
\end{minipage} & \begin{minipage}[b]{0.30\columnwidth}\centering
Identity\strut
\end{minipage}\tabularnewline
\midrule
\endhead
\begin{minipage}[t]{0.30\columnwidth}\centering
\(\sqrt{a^2-x^2}\)\strut
\end{minipage} & \begin{minipage}[t]{0.30\columnwidth}\centering
\(x = a\sin \theta, -\dfrac{\pi}{2} \leq \theta \leq \dfrac{\pi}{2}\)\strut
\end{minipage} & \begin{minipage}[t]{0.30\columnwidth}\centering
\(1 - \sin^2 \theta = \cos^2 \theta\)\strut
\end{minipage}\tabularnewline
\begin{minipage}[t]{0.30\columnwidth}\centering
\strut
\end{minipage} & \begin{minipage}[t]{0.30\columnwidth}\centering
\strut
\end{minipage} & \begin{minipage}[t]{0.30\columnwidth}\centering
\strut
\end{minipage}\tabularnewline
\begin{minipage}[t]{0.30\columnwidth}\centering
\(\sqrt{a^2+x^2}\)\strut
\end{minipage} & \begin{minipage}[t]{0.30\columnwidth}\centering
\(x = a\tan \theta, -\dfrac{\pi}{2} \leq \theta \leq \dfrac{\pi}{2}\)\strut
\end{minipage} & \begin{minipage}[t]{0.30\columnwidth}\centering
\(1 - \tan^2 \theta = \sec^2 \theta\)\strut
\end{minipage}\tabularnewline
\begin{minipage}[t]{0.30\columnwidth}\centering
\strut
\end{minipage} & \begin{minipage}[t]{0.30\columnwidth}\centering
\strut
\end{minipage} & \begin{minipage}[t]{0.30\columnwidth}\centering
\strut
\end{minipage}\tabularnewline
\begin{minipage}[t]{0.30\columnwidth}\centering
\(\sqrt{x^2-a^2}\)\strut
\end{minipage} & \begin{minipage}[t]{0.30\columnwidth}\centering
\(x = a \sec \theta, 0 \leq \theta \leq \dfrac{\pi}{2}\) or
\(\pi \leq \theta \leq \dfrac{3\pi}{2}\)\strut
\end{minipage} & \begin{minipage}[t]{0.30\columnwidth}\centering
\(\sec^2 \theta - 1 = \tan^2 \theta\)\strut
\end{minipage}\tabularnewline
\begin{minipage}[t]{0.30\columnwidth}\centering
\strut
\end{minipage} & \begin{minipage}[t]{0.30\columnwidth}\centering
\strut
\end{minipage} & \begin{minipage}[t]{0.30\columnwidth}\centering
\strut
\end{minipage}\tabularnewline
\bottomrule
\end{longtable}

\begin{center}\rule{0.5\linewidth}{0.5pt}\end{center}

\begin{multicols}{3}

\hypertarget{partial-fractions}{%
\subsection{Partial fractions}\label{partial-fractions}}

Let \(f(x) = \dfrac{P(x)}{Q(x)}\) where \(P\),\(Q\) are polynomial
functions.

If \(\deg(P) \leq \deg(Q)\), then continue doing partial fractions.

If \(\deg(P) \geq \deg(Q)\), then we need to do preliminary work:

\[ f(x) = S(x) + \frac{R(x)}{Q(x)} \]

where \(S\) is a polynomial function and \(R\) is the remainder of the
long division between \(P\) and \(Q\).

\begin{center}\rule{0.5\linewidth}{0.5pt}\end{center}

\hypertarget{arc-length}{%
\subsection{Arc length}\label{arc-length}}

If \(f'\) is continuous on \([a,b]\), then the length \(L\) of the curve
\(y = f(x)\), \(a \leq x \leq b\), is given by

\[ L = \int_a^b \sqrt{1+[f'(x)]^2}dx \]

\begin{center}\rule{0.5\linewidth}{0.5pt}\end{center}

\hypertarget{simple-growth-model}{%
\subsection{Simple growth model}\label{simple-growth-model}}

The solution of the initial-value problem

\[ \frac{dP}{dt} = kP, P(0) = P_0 \]

is

\[P(t) = P_0e^{kt} \]

\begin{center}\rule{0.5\linewidth}{0.5pt}\end{center}

\hypertarget{logistic-model}{%
\subsection{Logistic model}\label{logistic-model}}

The solution fo the initial-value problem

\[\dfrac{\dfrac{dP}{dt}}{P} = k \left(1-\frac{P}{M}\right)\]

is

\[ P(t) = \frac{M}{1-Ae^{kt}}, A = \frac{M-P_0}{P_0} \]


\end{multicols}

\end{document}
