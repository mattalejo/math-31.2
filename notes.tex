% Options for packages loaded elsewhere
\PassOptionsToPackage{unicode}{hyperref}
\PassOptionsToPackage{hyphens}{url}
%
\documentclass[
]{article}
\usepackage{lmodern}
\usepackage{amssymb,amsmath}
\usepackage{ifxetex,ifluatex}
\ifnum 0\ifxetex 1\fi\ifluatex 1\fi=0 % if pdftex
  \usepackage[T1]{fontenc}
  \usepackage[utf8]{inputenc}
  \usepackage{textcomp} % provide euro and other symbols
\else % if luatex or xetex
  \usepackage{unicode-math}
  \defaultfontfeatures{Scale=MatchLowercase}
  \defaultfontfeatures[\rmfamily]{Ligatures=TeX,Scale=1}
\fi
% Use upquote if available, for straight quotes in verbatim environments
\IfFileExists{upquote.sty}{\usepackage{upquote}}{}
\IfFileExists{microtype.sty}{% use microtype if available
  \usepackage[]{microtype}
  \UseMicrotypeSet[protrusion]{basicmath} % disable protrusion for tt fonts
}{}
\makeatletter
\@ifundefined{KOMAClassName}{% if non-KOMA class
  \IfFileExists{parskip.sty}{%
    \usepackage{parskip}
  }{% else
    \setlength{\parindent}{0pt}
    \setlength{\parskip}{6pt plus 2pt minus 1pt}}
}{% if KOMA class
  \KOMAoptions{parskip=half}}
\makeatother
\usepackage{xcolor}
\IfFileExists{xurl.sty}{\usepackage{xurl}}{} % add URL line breaks if available
\IfFileExists{bookmark.sty}{\usepackage{bookmark}}{\usepackage{hyperref}}
\hypersetup{
  pdftitle={Mathematical Analysis IB},
  hidelinks,
  pdfcreator={LaTeX via pandoc}}
\urlstyle{same} % disable monospaced font for URLs
\usepackage[margin=1in]{geometry}
\usepackage{longtable,booktabs}
% Correct order of tables after \paragraph or \subparagraph
\usepackage{etoolbox}
\makeatletter
\patchcmd\longtable{\par}{\if@noskipsec\mbox{}\fi\par}{}{}
\makeatother
% Allow footnotes in longtable head/foot
\IfFileExists{footnotehyper.sty}{\usepackage{footnotehyper}}{\usepackage{footnote}}
\makesavenoteenv{longtable}
\usepackage{graphicx}
\makeatletter
\def\maxwidth{\ifdim\Gin@nat@width>\linewidth\linewidth\else\Gin@nat@width\fi}
\def\maxheight{\ifdim\Gin@nat@height>\textheight\textheight\else\Gin@nat@height\fi}
\makeatother
% Scale images if necessary, so that they will not overflow the page
% margins by default, and it is still possible to overwrite the defaults
% using explicit options in \includegraphics[width, height, ...]{}
\setkeys{Gin}{width=\maxwidth,height=\maxheight,keepaspectratio}
% Set default figure placement to htbp
\makeatletter
\def\fps@figure{htbp}
\makeatother
\setlength{\emergencystretch}{3em} % prevent overfull lines
\providecommand{\tightlist}{%
  \setlength{\itemsep}{0pt}\setlength{\parskip}{0pt}}
\setcounter{secnumdepth}{-\maxdimen} % remove section numbering
\ifluatex
  \usepackage{selnolig}  % disable illegal ligatures
\fi

\title{Mathematical Analysis IB}
\author{}
\date{\vspace{-2.5em}}

\begin{document}
\maketitle

\begin{quote}
\href{../math-31.2/notes.pdf}{Download the PDF copy of the notes here}
\end{quote}

\begin{quote}
\href{../math-31.2/formulas.pdf}{Download the formula sheet here}
\end{quote}

\hypertarget{review-on-differentiation}{%
\section*{Review on differentiation}\label{review-on-differentiation}}
\addcontentsline{toc}{section}{Review on differentiation}

\hypertarget{differentiability}{%
\subsection{Differentiability}\label{differentiability}}

Let \(f\) be a function on some open interval \(I\) containing \(x\).
The derivative of \(f\) at \(x\), denoted by \(f'(x)\), is

\[ f'(x) = \lim_{h\to 0}\frac{f(x+h)-f(x)}{h} \]

\begin{center}\rule{0.5\linewidth}{0.5pt}\end{center}

\hypertarget{differentiation-rules}{%
\subsubsection{Differentiation rules}\label{differentiation-rules}}

\begin{enumerate}
\def\labelenumi{\arabic{enumi}.}
\item
  \(\displaystyle\frac{d}{dx}(cf(x))= cf'(x)\)
\item
  \(\displaystyle\frac{d}{dx}(f(x) \pm g(x)) = f'(x) \pm g'(x)\)
\item
  \(\displaystyle\frac{d}{dx}(f(x)g(x)) = f(x)g'(x) + g(x)f'(x)\)
\item
  \(\displaystyle\frac{d}{dx}\Big(\frac{f(x)}{g(x)}\Big) = \frac{g(x)f'(x)-f(x)g'(x)}{(g(x))^2}\)
\item
  \(\displaystyle\frac{d}{dx}(f(g(x))) = f'(g(x))g'(x)\)
\end{enumerate}

\hypertarget{differentiation-formulas-i}{%
\subsubsection{Differentiation formulas
I}\label{differentiation-formulas-i}}

\begin{enumerate}
\def\labelenumi{\arabic{enumi}.}
\item
  \(\displaystyle\frac{d}{dx}(c)=0, c \in \mathbb{R}\)
\item
  \(\displaystyle\frac{d}{dx}(x^r)=rx^{r-1}, r \in \mathbb{R}\)
\item
  \(\displaystyle\frac{d}{dx}(\sin x)=\cos x\)
\item
  \(\displaystyle\frac{d}{dx}(\cos x)=\sin x\)
\item
  \(\displaystyle\frac{d}{dx}(\tan x)=\sec^2x\)
\item
  \(\displaystyle\frac{d}{dx}(\cot x)=-\csc^2x\)
\item
  \(\displaystyle\frac{d}{dx}(\sec x) = \sec x\tan x\)
\item
  \(\displaystyle\frac{d}{dx}(\csc x)=-\csc x\cot x\)
\end{enumerate}

\hypertarget{differentiation-formulas-ii}{%
\subsubsection{Differentiation formulas
II}\label{differentiation-formulas-ii}}

\begin{enumerate}
\def\labelenumi{\arabic{enumi}.}
\item
  \(\displaystyle\frac{d}{dx}(e^x) = e^x\)
\item
  \(\displaystyle\frac{d}{dx}(\ln|x|) = \frac{1}{x}\)
\item
  \(\displaystyle\frac{d}{dx}(\sin^{-1}x) = \frac{1}{\sqrt{1-x^2}}\)
\item
  \(\displaystyle\frac{d}{dx}(\tan^{-1}x) = \frac{1}{1+x^2}\)
\item
  \(\displaystyle\frac{d}{dx}(\sec^{-1}x) = \frac{1}{x \sqrt{x^2-1}}\)
\end{enumerate}

\begin{center}\rule{0.5\linewidth}{0.5pt}\end{center}

\hypertarget{mean-value-theorem}{%
\subsection{Mean value theorem}\label{mean-value-theorem}}

Let \(f\) be a function that is continuous on \([a,b]\) and is
differentiable on \((a,b)\). Then there is a number \(c\in(a,b)\) such
that

\[ f'(c)=\frac{f(b)-f(a)}{b-a} \]

\hypertarget{consequences-of-mvt}{%
\subsection{Consequences of MVT}\label{consequences-of-mvt}}

\hypertarget{zero-derivative}{%
\subsubsection{Zero derivative}\label{zero-derivative}}

If \(f'(x)=0 \;\forall x\) in interval \(I\), then
\(f(x)=c \;\forall x\in I\) for some constant \(C\).

\hypertarget{equal-derivatives}{%
\subsubsection{Equal derivatives}\label{equal-derivatives}}

If \(f'(x)-g'(x)=0 \;\forall x\) in an interval \(I\), then
\(f(x)=g(x)+C\) for some constant \(C\).

\hypertarget{example}{%
\paragraph*{Example}\label{example}}
\addcontentsline{toc}{paragraph}{Example}

\begin{quote}
Let \(f(x)=\cos^{-1}x\) and \(g(x)=-\sin^{-1}x\).

This implies that \(x \in [-1,1]\) and
\(\displaystyle f(x),g(x) \in \Big[-\frac{\pi}{2},\frac{\pi}{2}\Big]\)

\[ f'(x)=-\frac{1}{\sqrt{x^2+1}}\]

\[ g'(x)=-\frac{1}{\sqrt{x^2+1}} \] Since \(f'(x)-g'(x)=0\) for
\(x \in [-1,1]\), then \(f(x)-g(x)=C\) for some constant \(C\) by a
corollary.

\begin{align*}
\cos^{-1}x - (-\sin^{-1}x)&=C \\
\cos^{-1}x +\sin^{-1}x&=C
\end{align*}

Substituting \(x \in [-1,1]\), in this case, let's use \(x=0\),

\begin{align*}
\cos^{-1}(0) +\sin^{-1}(0)&=C \\
0 + \frac{\pi}{2} &= C \\
C &= \frac{\pi}{2} 
\end{align*}

\[ \therefore \forall x \in[-1,1],f(x)-g(x)= \frac{\pi}{2}\]
\end{quote}

\begin{center}\rule{0.5\linewidth}{0.5pt}\end{center}

\hypertarget{differentials}{%
\subsection{Differentials}\label{differentials}}

\begin{align*}
f'(x)   &= \frac{dy}{dx}\\
f'(x)dx &= dy
\end{align*}

\begin{center}\rule{0.5\linewidth}{0.5pt}\end{center}

\hypertarget{indefinite-and-definite-integrals}{%
\section{Indefinite and definite
integrals}\label{indefinite-and-definite-integrals}}

\hypertarget{indefinite-integral}{%
\subsection{Indefinite integral}\label{indefinite-integral}}

The main interpretation of derivative is the slope of a tangent line of
a curve.

\hypertarget{example-1}{%
\paragraph*{Example}\label{example-1}}
\addcontentsline{toc}{paragraph}{Example}

\begin{quote}
At any point \((x,y)\) on a particular curve \(y=F(x)\), the tangent
line has a slope equal to \(4x-5\). If the curve contains the point
\((3,7)\), find \(F(x)\).

\textbf{Solution.} Since the slope is equal to \(4x-5\) for any point
\((x,y)\), then the slope at \((3,7)\) is \(4(3)-5=7\).

\(4x-5\) therefore represents the tangent slope for all values of \(x\).
So

\[ F'(x)=4x-5 \]

By intuition, we can conclude that \(F(x)=2x^2-5x\).

However given \(F(x)=2x^2-5x+1\), \(F'(x)\) remains the same. And so is
\(F(x)=2x^2-5x-3\), \(F(x)=2x^2-5x+\pi\), and infinitely more functions.
We can arbitrarily assign a constant \(k\), so that \(F(x)=2x^2-5x+k\).

Substituting \((x,y)=(3,7),\)

\begin{align*}
7 &= 2(3)^2 - 5(3) +k\\
7 &= 18 - 15 + k \\
k &= 4 
\end{align*}

So \(F(x)=2x^2-5x+4\).
\end{quote}

\hypertarget{definition-of-an-antiderivative}{%
\subsubsection{Definition of an
antiderivative}\label{definition-of-an-antiderivative}}

A function \(F\) is called an antiderivative of the function \(f\) on an
interval \(I\) if \(F'(x) = f(x) \;\forall x \in I\).

\(F(x)=2x^2-5x\) is a \textbf{possible} antiderivative of \(f(x)=4x-5\).
\(F(x)=2x^2-5x+4\) is also a \textbf{possible} antiderivative of
\(f(x)=4x-5\).

\hypertarget{equal-derivatives-1}{%
\subsubsection{Equal derivatives}\label{equal-derivatives-1}}

If \(F'(x)=G'(x) \;\forall x\) in an interval \(I\), then
\(F(x) = G(x) + C \;\forall x \in I\) for some constant \(C\).

\hypertarget{integration-notation}{%
\subsubsection{Integration notation}\label{integration-notation}}

The collection of all antiderivatives of \(f\) is denoted by

\[ \int f(x)dx \]

which is read as ``the integral of \(f(x)dx\).''

This collection is also called the \textbf{indefinite integral} of
\(f\).

The reverse process if differentiation is called
\textbf{antidifferentiation} or \textbf{integration}.

\(\displaystyle\int (4x-5)dx = 2x^2-5x+C\) for some constant \(C\).

\(C\) is the constant of integration.

\(\displaystyle\int \sin xdx = -\cos x+C\)

\begin{center}\rule{0.5\linewidth}{0.5pt}\end{center}

\hypertarget{integration-rules}{%
\subsubsection{Integration rules}\label{integration-rules}}

\begin{enumerate}
\def\labelenumi{\arabic{enumi}.}
\item
  \(\displaystyle\int kf(x)dx = k \int f(x)dx\), \(k \in \mathbb{R}\)
\item
  \(\displaystyle\int f(x) \pm g(x) dx = \int f(x)dx \pm \int g(x)dx\)
\end{enumerate}

\hypertarget{integration-formulas-i}{%
\subsubsection{Integration formulas I}\label{integration-formulas-i}}

\begin{enumerate}
\def\labelenumi{\arabic{enumi}.}
\item
  \(\displaystyle\int kdx = kx + C, k \in \mathbb{R}\)
\item
  \(\displaystyle\int x^ndx = \frac{x^{n+1}}{n+1}+C, n \in \mathbb{R}, n \neq -1\)
\end{enumerate}

\hypertarget{integration-formulas-ii}{%
\subsubsection{Integration formulas II}\label{integration-formulas-ii}}

\begin{enumerate}
\def\labelenumi{\arabic{enumi}.}
\item
  \(\displaystyle\int \sin xdx = -\cos x + C\)
\item
  \(\displaystyle\int \cos xdx = \sin x + C\)
\item
  \(\displaystyle\int \sec^2xdx = \tan x + C\)
\item
  \(\displaystyle\int \csc^2xdx = -\cot x + C\)
\item
  \(\displaystyle\int \sec x\tan xdx = \sec x + C\)
\item
  \(\displaystyle\int \csc x\cot xdx = -\csc x +C\)
\end{enumerate}

\hypertarget{integration-formulas-iii}{%
\subsubsection{Integration formulas
III}\label{integration-formulas-iii}}

\begin{enumerate}
\def\labelenumi{\arabic{enumi}.}
\item
  \(\displaystyle\int e^xdx =e^x +C\)
\item
  \(\displaystyle\int \frac{1}{x}dx=\ln|x|+C\)
\item
  \(\displaystyle\int \frac{1}{\sqrt{1-x^2}}dx = \sin^{-1}x +C\)
\item
  \(\displaystyle\int \frac{1}{1+x^2}dx = \tan^{-1}x +C\)
\item
  \(\displaystyle\int \frac{1}{x\sqrt{x^2-1}}dx= \sec^{-1}+C\)
\end{enumerate}

\begin{center}\rule{0.5\linewidth}{0.5pt}\end{center}

\hypertarget{substitution-rule}{%
\subsection{Substitution rule}\label{substitution-rule}}

\hypertarget{chain-rule-for-derivatives}{%
\subsubsection{Chain rule for
derivatives}\label{chain-rule-for-derivatives}}

\[ \frac{d}{dx}(f(g(x)))=f'(g(x))g'(x) \]

If follows that

\[ \int f'(g(x))g'(x)dx = f(g(x))+C \]

\hypertarget{example-2}{%
\paragraph*{Example}\label{example-2}}
\addcontentsline{toc}{paragraph}{Example}

\begin{quote}
Evaluate \(\int 2x\cos x^2dx\).

\textbf{Preliminary work.} By intuition, we can get \(f(x)=sinx\) and
\(g(x)=x^2\)

\[ \int 2x\cos x^2dx = f(g(x)) = \sin x^2 \]

\textbf{Solution.} Suppose that \(f'(x)=\frac{dy}{dx}\)

\[ dy = f'(x)dx \]

Let \(u=g(x)\), then \(g'(x)=\frac{du}{dx}\)

\[ du = g'(x)dx \]

Let \(u=x^2\)

\begin{align*}
du &=2xdx\\
 \int 2x\cos x^2dx &= \int \cos udu\\
 &=\sin u+C \\
 &=\sin x^2+C
\end{align*}
\end{quote}

\hypertarget{definition-of-the-substitution-rule}{%
\subsubsection{Definition of the substitution
rule}\label{definition-of-the-substitution-rule}}

If \(u=g(x)\) is a differentiable function whose range is interval \(I\)
and \(f\) is continuous on \(I\), then

\[ \int f'(g(x))g'(x)dx = \int f(u)du \]

\begin{center}\rule{0.5\linewidth}{0.5pt}\end{center}

\hypertarget{definite-integrals}{%
\subsection{Definite integrals}\label{definite-integrals}}

\hypertarget{the-area-problem}{%
\subsubsection{The area problem}\label{the-area-problem}}

Let \(f\) be a continuous nonnegative function on \([a,b]\). Find the
area of the regiom bounded by the curve \(y=f(x)\), the lines \(x=a\),
\(x=b\), and the \(x\)-axis.

\begin{quote}
The area is often coined the \textbf{region under the curve}, which
generally means the area in between the curve and the \(x\)-axis
\end{quote}

\hypertarget{example-3}{%
\paragraph*{Example}\label{example-3}}
\addcontentsline{toc}{paragraph}{Example}

\begin{quote}
Consider \(f(x) = x^2 +1\) on \([0,2]\).

\textbf{Solution.} Let \(A\) be the area under the curve

Using right endpoints (5 rectangles)

\[ \Delta x = \frac{2-0}{5} = \frac{2}{5} = 0.4 \] Rectangle 1:
\((\Delta x)(f(0.4)) = (0.4)(1.16)\)

Rectangle 2: \((\Delta x)(f(0.8)) = (0.4)(1.64)\)

Rectangle 3: \((\Delta x)(f(1.2)) = (0.4)(2.44)\)

Rectangle 4: \((\Delta x)(f(1.6)) = (0.4)(3.56)\)

Rectangle 5: \((\Delta x)(f(2.0)) = (0.4)(5)\)

\[ A_5^+ = (0.4)(1.16) + (0.4)(1.64) + (0.4)(2.44) + (0.4)(3.56) +(0.4)(5) = 5.52 \]
\(A_5^+\) is an overestimation of \(A\).

Using left endpoints (5 rectangles):

Rectangle 1: \((\Delta x)(f(0)) = (0.4)(1)\)

Rectangle 2: \((\Delta x)(f(0.4)) = (0.4)(1.16)\)

Rectangle 3: \((\Delta x)(f(0.8)) = (0.4)(1.64)\)

Rectangle 4: \((\Delta x)(f(1.2)) = (0.4)(2.44)\)

Rectangle 5: \((\Delta x)(f(1.6)) = (0.4)(3.56)\)

\[ A_5^- = (0.4)(1) + (0.4)(1.16) + (0.4)(1.64) + (0.4)(2.44) + (0.4)(3.56) + =3.92 \]

\(A_5^-\) is an underestimation of \(A\).

We can increase the number of rectangles and compute the area \(A\) more
\textbf{accurately} by computing the area as the number of rectangles
approach infinity.

Let the number of rectangles be \(n\)

\[ \Delta x = \frac{2-0}{n}=\frac{2}{n} \] Let \(x_0\) be the first
point: \(x_0=0\)

\begin{align*}
x_1 &= \frac{2}{n} & x_2 &= \frac{4}{n} & x_3 &= \frac{6}{n}\\
x_4 &= \frac{8}{n} & x_5 &= \frac{10}{n} & x_6 &= \frac{12}{n}\\
x_7 &= \frac{14}{n} &    &\cdots      & x_i &= \frac{2i}{n}
\end{align*}\\

\begin{align*}
A_n &= R_1 + R_2 + R_3 + R_4 + \cdots + R_n\\
&= \sum_{i=1}^n \Delta x (f(x_i))\\
&= \sum_{i=1}^n \frac{2}{n} \Big[\Big(\frac{2i}{n}\Big)^2 +1\Big]\\
&= \frac{2}{n} \sum_{i=1}^n \Big(\frac{4i^2}{n^2} +1\Big)\\
&= \frac{2}{n} \Big[\sum_{i=1}^n \Big(\frac{4i^2}{n^2}\Big) + \sum_{i=1}^n(1)\Big]\\
&= \frac{2}{n} \Big[\frac{4}{n^2}\sum_{i=1}^n (i^2) + \sum_{i=1}^n(1)\Big]\\
&= \frac{2}{n} \Big[\frac{4}{n^2} \Big(\frac{(n)(n+1)(2n+1)}{6}\Big)+n\Big]\\
&= \frac{8}{n^3} \frac{(n)(n+1)(2n+1)}{6} +2\\
A_n &= \frac{4}{3}\Big(1+\frac{1}{n}\Big)\Big(2+\frac{1}{n}\Big) +2
\end{align*}\\

\begin{align*}
A &= \lim_{n \to\infty} A_n\\
&= \lim_{n \to \infty}  \sum_{i=1}^n \frac{2}{n} \Big[\Big(\frac{2i}{n}\Big)^2 +1\Big]\\
&= \lim_{n \to \infty} \Big[ \frac{4}{3}\Big(1+\frac{1}{n}\Big)\Big(2+\frac{1}{n}\Big) +2\Big]\\
&= \frac{4}{3}(1)(2)+ 2\\
A &= \frac{14}{3}
\end{align*}
\end{quote}

\hypertarget{riemann-sum}{%
\subsubsection{Riemann sum}\label{riemann-sum}}

\begin{quote}
Let \(f\) be a function defined on \([a,b]\).

Divide \([a,b]\) into \(n\) subintervals, each with width

\[ \Delta x = \frac{b-a}{n} \]

Let \(x_0=a, x_1,x_2, \ldots, x_n = b\),

For each subinterval \([x_{i-1},x_i]\), choose a sample point \(x_i^*\)

Compute the sum

\[A_n = \sum_{i=1}^{n}f(x_i^*)\Delta x \]

This is also called the \textbf{Riemann sum}.
\end{quote}

\[A_n = \sum_{i=1}^{n}f(x_i^*)\Delta x \]

\hypertarget{definite-integral-and-integrability}{%
\subsubsection{Definite integral and
integrability}\label{definite-integral-and-integrability}}

The definite integral of \(f\) from \(a\) to \(b\) is

\[ \int_a^b f(x)dx = \lim_{x\to\infty}\sum_{i=1}^{n} f(x_i^*)\Delta x \]
provided that such limit exists.

We say that \(f\) is integrable on \([a,b]\)

\begin{center}\rule{0.5\linewidth}{0.5pt}\end{center}

\hypertarget{remarks-on-the-definite-integral}{%
\subsubsection{Remarks on the definite
integral}\label{remarks-on-the-definite-integral}}

\begin{enumerate}
\def\labelenumi{\arabic{enumi}.}
\item
  If a function is continuous on \([a,b]\), it is integrable on
  \([a,b]\).
\item
  If \(f\) is a nonnegative continuous function on \([a,b]\), then
  \(\displaystyle\int_a^b f(x)dx\) is the area under the curve
  \(y=f(x)\) from \(x=a\) and \(x=b\)
\item
  \(\displaystyle\int_a^b f(x)dx = \int_a^b f(y)dy\)
\end{enumerate}

\hypertarget{conventions-on-the-definite-integral}{%
\subsubsection{Conventions on the definite
integral}\label{conventions-on-the-definite-integral}}

\begin{enumerate}
\def\labelenumi{\arabic{enumi}.}
\item
  \(\displaystyle\int_b^a f(x)dx = -\int_a^b f(x)dx\)
\item
  \(\displaystyle\int_a^a f(x)dx = 0\)
\end{enumerate}

\hypertarget{properties-of-the-definite-integral}{%
\subsubsection{Properties of the definite
integral}\label{properties-of-the-definite-integral}}

\begin{enumerate}
\def\labelenumi{\arabic{enumi}.}
\item
  \(\displaystyle\int_a^b cf(x)dx = c\int_a^b f(x)dx\)
\item
  \(\displaystyle\int_a^b [f(x) \pm g(x)] dx = \int_a^b f(x) \pm \int_a^b g(x)\)
\item
  \(\displaystyle\int_a^c f(x)dx + \int_c^b f(x)dx = \int_a^b f(x)dx\)
\item
  If \(f(x)\geq0 \;\forall x \in[a,b]\), then
  \(\displaystyle\int_a^b f(x)dx \geq 0\)
\item
  If \(f(x) \geq g(x)\;\forall x\in[a,b]\), then
  \(\displaystyle\int_a^b f(x)dx \geq \int_a^b g(x)dx\)
\item
  If \(m \leq f(x) \leq M \;\forall x\in[a,b]\), then
  \(\displaystyle m(b-a) \leq \int_a^b f(x)dx \leq M(b-a)\)
\end{enumerate}

\begin{center}\rule{0.5\linewidth}{0.5pt}\end{center}

\hypertarget{the-fundamental-theorem-of-calculus}{%
\subsection{The Fundamental Theorem of
Calculus}\label{the-fundamental-theorem-of-calculus}}

\hypertarget{mean-value-theorem-for-integrals}{%
\subsubsection{Mean Value Theorem for
integrals}\label{mean-value-theorem-for-integrals}}

\begin{quote}
Proof

Since \(f\) is continuous on \([a,b]\), then \(f\) is integrable on
\([a,b]\) --- i.e.~\(\int_a^b f(x)dx\) has a value.

Since f is continuous on {[}a,b{]}, by the \textbf{Extreme Value
Theorem}, \(\exists m, M \in \mathbb{R}\) such that
\(f(x_m) = m, f(x_M) = M, m \leq f(x) \leq M \;\forall x \in [a,b]\) and
for some \(x_m, x_M \in [a,b]\).

By Property 6 of the definite integral,
\(\displaystyle m(b-a) \leq \int_a^b f(x)dx \leq M(b-a)\)

\begin{align*}
m      &\leq \frac{\int_a^b f(x)dx}{b-a} \leq& M \\
f(x_m) &\leq \frac{\int_a^b f(x)dx}{b-a} \leq& f(x_M)  \\
\end{align*}

By the IVT, \(\exists c \in [a,b]\) such that

\begin{align*}
\frac{\int_a^b f(x)dx}{b-a} &= f(c)\\
\int_a^b f(x)dx &= f(c)(b-a)
\end{align*}
\end{quote}

If \(f\) is continuous on \([a,b]\), \(\exists c \in [a,b]\) such that

\[ \int_b^a f(x)dx = f(c)(b-a) \]

\hypertarget{average-value-of-a-function}{%
\subsubsection{Average value of a
function}\label{average-value-of-a-function}}

\begin{quote}
Proof

Given a function continuous on \([a,b]\), we can get the average value
of the function at \([a,b]\) by dividing the curve into \(n\)
equal-width rectangles, getting the value of each sample points, and
dividing by \(n\).

Average area \(\displaystyle = \frac{\sum_{i=1}^n f(x_i^*)i}{n}\)

But then,
\(\displaystyle \Delta x = \frac{b-a}{n} \implies n = \frac{b-a}{\Delta x}\)

\begin{align*}
\frac{\sum_{i=1}^n f(x_i^*)}{n} &= \frac{\sum_{i=1}^n f(x_i^*)}{\frac{b-a}{\Delta x}}\\
&= \frac{1}{b-a}\sum_{i=1}^n f(x_i^*) \Delta x
\end{align*}

We want to make \(n\) larger in order to make the average more accurate.

\begin{align*}
\lim_{n\to\infty}\frac{\sum_{i=1}^n f(x_i^*)}{n} &= \lim_{n\to\infty}\frac{1}{b-a}\sum_{i=1}^n f(x_i^*)i \Delta x\\
&= \frac{1}{b-a}\lim_{n\to\infty}\sum_{i=1}^n f(x_i^*) \Delta x\\
&= \frac{1}{b-a}\int_a^b f(x)dx
\end{align*}

Therefore, given function \(f\) that is continuous on \([a,b]\), there
exists \(c \in [a,b]\) such that

\[ f_{avg} = f(c) \]
\end{quote}

Let \(f\) be a continuous on \([a,b]\). The average value of \(f\) at
\([a,b]\), denoted by \(f_{avg}\) is

\[ f_{avg} = \frac{\int_a^b f(x)dx}{b-a} \]

\begin{center}\rule{0.5\linewidth}{0.5pt}\end{center}

\hypertarget{first-part-of-the-fundamental-theorem-of-calculus}{%
\subsubsection{First part of the Fundamental Theorem of
Calculus}\label{first-part-of-the-fundamental-theorem-of-calculus}}

Let \(y=f(t)\) that is continuous on \([a,b]\).

If \(x \in [a,b]\), then the function is also continuous on
\([a,b] \implies\) the function is also continuous on \([a,x]\).

\begin{align*}
F(x) &= \int_a^x f(t)dt \\
F(a) &= \int_a^a f(t)dt = 0 \\
F(b) &= \int_a^b f(t)dt
\end{align*}

Let \(f\) be continuous on \([a,b]\). If f is the function defined by

\[ F(x) = \int_a^x f(t)dt \]

then \(F'(x) = f(x) \;\forall x \in [a,b]\).

\begin{quote}
Proof

Let \(x, x+h \in [a,b], h\neq 0\).

\[ F(x+h)-F(x) = \int_a^{x+h} f(t)dt -\int_a^x f(t)dt \]

By the Property 3 of definite integrals, \begin{align*}
\int_a^{x+h} f(t)dt -\int_a^x f(t)dt &= \int_a^x f(t)dt + \int_x^{x+h}f(t)dt -\int_a^x f(t)dt\\
&= \int_x^{x+h}f(t)dt
\end{align*}

By the Mean Value Theorem for integrals, \(\exists c \in [x,x+h]\) such
that

\begin{align*}
\int_x^{x+h}f(t)dt &= f(c)(x+h-x)\\
&=hf(c)\\
\implies F(x+h)-F(x) &= hf(c)\\
\frac{F(x+h)-F(x)}{h} &= f(c)\\
\lim_{h\to 0}\frac{F(x+h)-F(x)}{h} &= \lim_{h\to 0}f(c)
\end{align*}

Note that \(\displaystyle \lim_{h\to 0}x = x\) and
\(\displaystyle \lim_{h\to 0}(x+h) = x \implies \lim_{h\to 0}c = x\) by
Squeeze Theorem.

Since \(f\) is continuous at \(x\),

\[ \lim_{h\to 0}f(c) = f(x) \]

\begin{align*}
\implies F'(x) &= f(x) \;\forall x \in [a,b]\\
\frac{d}{dx}\int_a^x f(t)dt &= f(x)
\end{align*}
\end{quote}

\begin{center}\rule{0.5\linewidth}{0.5pt}\end{center}

\hypertarget{second-part-of-the-fundamental-theorem-of-calculus}{%
\subsubsection{Second part of the Fundamental Theorem of
Calculus}\label{second-part-of-the-fundamental-theorem-of-calculus}}

\begin{quote}
Let's bring back \(f(x)=x^2+1\) on \([0,2]\).

\(f\) is continuous on \([0,2] \implies f\) is integrable on \([0,2]\).

\[ \implies \int_0^2 (x^2+1)dx = \frac{14}{3} \]

Let \(\displaystyle F(x)= \frac{x^3}{3}+ x-1\).

\begin{align*}
F(2) &= \frac{2^3}{3}+2-1 = \frac{8}{3}+1 = \frac{11}{3}\\
F(0) &= \frac{0^3}{3}+0-1 = 0-1 = - 1\\
F(2) - F(0) &= \frac{11}{3} -(-1) = \frac{14}{3} \\
\int_0^2 (x^2+1)dx &= F(2)-F(0)
\end{align*}

Observe that \(F'(x) = x^2 +1 \implies F(x)\) is the an antiderivative
of \(x^2+1\).
\end{quote}

If a function \(f\) is continuous on \([a,b]\), then

\[ \int_a^b f(x)dx = F(b)-F(a) \]

where \(F\) is any antiderivative of \(f\) on \([a,b]\).

\begin{quote}
The following notations for \(F(b)-F(a)\) are very useful in evaluating
definite integrals:

\(\displaystyle F(x)\Big]_a^b\) or \(\displaystyle F(x)\Big|_a^b\)
\end{quote}

\begin{quote}
Proof

By FTC - Part 1, the function

\[ \int_a^x f(t)dt \]

is an antiderivative of f on \([a,b]\).

By the Equal Derivatives Theorem,

\[ \int_a^x f(t)dt = F(x) + C \]

where \(F\) is any antiderivative of \(f\).

\begin{align*}
 x= b, \int_a^b f(t)dt &= F(b) + C \\
 x=a , \int_a^a f(t)dt &= F(a) + C = 0 \\
\int_a^b f(t)dt - \int_a^a f(t)dt &= [ F(b)+C ] - [F(a)+C] \\
\int_a^b f(t)dt &= F(b)-F(a)
\end{align*}
\end{quote}

\begin{center}\rule{0.5\linewidth}{0.5pt}\end{center}

\hypertarget{application-i}{%
\section{Application I}\label{application-i}}

\hypertarget{areas-between-curves}{%
\subsection{Areas between curves}\label{areas-between-curves}}

\hypertarget{example-1-1}{%
\paragraph*{Example 1}\label{example-1-1}}
\addcontentsline{toc}{paragraph}{Example 1}

\begin{quote}
Find the area of the region under the curve \(y=x^2-1\) from \(x=-1\) to
\(x=2\).

\textbf{Solution.} Area is simply not \(\int_{-1}^2 (x^2-1)dx\) because
\(\int_{-1}^1 (x^2-1)dx\) is negative and cancels the positive area.

Therefore, we get \(\int_{-1}^1 -(x^2-1)dx\) to get the area of the
curve between -1 and 1. \begin{align*}
A &= \int_{-1}^1-(x^2-1)dx+\int_1^2(x^2-1)dx \\
&= \Big(-\frac{x^3}{3}+x\Big)\Bigg|_{-1}^1+\Big(\frac{x^3}{3}-x\Big)\Bigg|_1^2\\
&= \Big(\frac{1^3}{3}+1\Big)-\Big[\frac{(-1)^3}{3}+(-1)\Big]+\Big(\frac{2^3}{3}-2\Big)-\Big(\frac{1^3}{3}-1\Big)\\
&= \frac{2}{3}+\frac{2}{3}+\frac{8}{3}-2+\frac{2}{3}\\
A &= \frac{8}{3} 
\end{align*}
\end{quote}

\hypertarget{example-2-1}{%
\paragraph*{Example 2}\label{example-2-1}}
\addcontentsline{toc}{paragraph}{Example 2}

\begin{quote}
Find the area of the region bounded by the curves of \(y=x^2\) and
\(y = 4x-x^2\).

\textbf{Solution.} Note that both curves intersect at \((0,0)\) and
\((2,4)\).

When we use Riemann sum, we only get the rectangles in between the
region bounded by the area by subtracting the upper function
(\(y=4x-x^2\)) to the lower function (\(y=x^2\))

\[ \implies A_n = \sum [(4x-x^2)-x^2]\Delta x \]

\begin{align*}
A &= \int_0^2[(4x-x^2)-x^2]dx\\
&= \int_0^2(4x-2x^2)dx\\
&= \Big(2x^2-\frac{2x^3}{3}\Big)\Bigg|_0^2\\
&= \Big[2(2)^2-\frac{2(2)^3}{3}\Big]-\Big[2(0)^2-\frac{2(0)^3}{3}\Big]\\
&= \Big[8-\frac{16}{3}\Big]-0\\
A &= \frac{8}{3}
\end{align*}
\end{quote}

\hypertarget{example-3-1}{%
\paragraph*{Example 3}\label{example-3-1}}
\addcontentsline{toc}{paragraph}{Example 3}

\begin{quote}
Find the area of the region bounded by the curve \(y=\sqrt{x}\), the
line \(x+2y =3\), and the \(x\)-axis.

\textbf{Solution.} The graphs intersect at \((0,0)\), \((1,1)\), and
\((3,0)\).

\[ x+2y = 3 \implies y  = -\frac{1}{2}x + \frac{3}{2} \]

\begin{align*}
A &= \int_0^1 (\sqrt{x})dx + \int_1^3 \Big(-\frac{1}{2}(3-x)\Big)dx\\
&= \Big(\frac{2x^{\frac{3}{2}}}{3}\Big) \Bigg|_0^1 + \Big(\frac{1}{2}(3x - \frac{x^2}{2})\Big)\Bigg|_1^3\\
&= \frac{2(1)^{\frac{3}{2}}}{3} - \frac{2(0)^{\frac{3}{2}}} + \frac{1}{2}(3(3) - \frac{3^2}{2}) - \frac{1}{2}(3(1) - \frac{1^2}{2})\\
&= \frac{2}{3}+ \frac{9}{4} - \frac{5}{4}\\
&= \frac{8-27+15}{12}\\
&= \frac{20}{12}\\
A &= \frac{5}{3}
\end{align*}
\end{quote}

\begin{center}\rule{0.5\linewidth}{0.5pt}\end{center}

\hypertarget{volumes-and-volumes-of-revolution-using-disks-and-washers}{%
\subsection{Volumes and volumes of revolution using disks and
washers}\label{volumes-and-volumes-of-revolution-using-disks-and-washers}}

\hypertarget{volume-of-a-right-cylinder}{%
\subsubsection{Volume of a right
cylinder}\label{volume-of-a-right-cylinder}}

\[ V = ah \]

\[ V_n = \sum_{i=1}^n A(x)\Delta x \]

Let \(S\) be a solid that lies between \(x=a\) and \(x=b\). If the
cross-sectional area of \(S\) in the plane \(P_x\) through \(x\) and
perpendicular to the \(x\)-axis is \(A(x)\), where \(A\) is a continuous
function on \([a,b]\), then the volume \(V\) of \(S\) is

\[ V = \lim_{n \to\infty} \sum_{i=1}^n A(x_i^*)\Delta x = \int_b^a A(x)dx \]

\hypertarget{example-1-2}{%
\paragraph*{Example 1}\label{example-1-2}}
\addcontentsline{toc}{paragraph}{Example 1}

\begin{quote}
Let us find the volume of a sphere of radius \(r\).

\textbf{Solution.}

radius of the cross-section circle at \(x = \sqrt{r^2-x^2}\)

\begin{align*}
A(x) &= \pi (\sqrt{r^2-x^2})^2\\
&= \pi (r^2-x^2)
\end{align*}

\begin{align*}
V_{\text{sphere}} &= \int_{-r}^r A(x)dx \\
&= \int_{-r}^r \pi (r^2-x^2) dx \\
&= \pi(r^2x - \frac{x^3}{3})\Big|_{-r}^r \\
&= \pi\Bigg[r^2(r) - \frac{r^3}{3}\Bigg] - \pi\Bigg[r^2(-r) - \frac{(-r)^3}{3}\Bigg] \\
V_{\text{sphere}} &= \frac{4}{3}\pi r^3
\end{align*}
\end{quote}

\hypertarget{example-2-2}{%
\paragraph*{Example 2}\label{example-2-2}}
\addcontentsline{toc}{paragraph}{Example 2}

\begin{quote}
The base of a solid is the region bounded by \(y=x^2\) and \(y=4\). Its
parallel cross-sections perpendicular to the base and the \(y\)-axis are
squares. Find the volume of the solid.

\textbf{Solution.} side of the cross-section at \(y = 2\sqrt{y}\)

\[ A(y) = (2\sqrt{y})^2 = 4y \]

\begin{align*}
V &= \int_0^4 A(y)dy \\
&=\int_0^4 4y dy \\
&= 2y^2 \Big|_0^4 \\
&= 2(4)^2 - 2(0)^2 \\
V &= 32
\end{align*}
\end{quote}

\begin{center}\rule{0.5\linewidth}{0.5pt}\end{center}

\hypertarget{volume-of-solids-of-revolution}{%
\subsubsection{Volume of solids of
revolution}\label{volume-of-solids-of-revolution}}

If we revolve a region about a line, we obtain a \textbf{solid of
revolution}.

\hypertarget{example-1-3}{%
\paragraph*{Example 1}\label{example-1-3}}
\addcontentsline{toc}{paragraph}{Example 1}

\begin{quote}
Consider the region under the curve \(y=x^2 +1\) from \(x=-1\) to
\(x=2\). We revolve this region about the \(x\)-axis.

\textbf{Solution.} radius of the cross-section at \(x = f(x)\)

\[ A(x) = \pi [f(x)]^2 \]

\begin{align*}
V &= \int_{-1}^2 \pi (x^2+1)^2 dx \\
&= \int_{-1}^2 \pi (x^4 +2x^2 +1) dx \\
&= \pi \Bigg(\frac{x^5}{5}+\frac{2x^3}{3}+x\Bigg) |_{-1}^2 \\
&= \pi\Bigg[\frac{2^5}{5}+\frac{2(2)^3}{3}+2\Bigg] - \pi\Bigg[\frac{(-1)^5}{5}+\frac{2(-1)^3}{3}+(-1)\Bigg] \\
V &= \frac{78\pi}{5}
\end{align*}
\end{quote}

The cross-section of a solid of revolution is always a circle.

\hypertarget{example-2-3}{%
\paragraph*{Example 2}\label{example-2-3}}
\addcontentsline{toc}{paragraph}{Example 2}

\begin{quote}
A solid is obtained by revolving about the \(x\)-axis the region bounded
by \(x=y^2\) and \(2y=x\). Find the volume of the solid.

\textbf{Solution.}

\begin{align*}
V &= \int_0^4 \pi(\sqrt{x})^2 dx - \int_0^4 \pi\Big(\frac{x}{2}\Big)^2 dx \\
&= \int_0^4 \pi (x) dx - \int_0^4 \pi \frac{x^4}{4} dx \\
V &= \frac{8\pi}{3}
\end{align*}
\end{quote}

\hypertarget{example-3-2}{%
\paragraph*{Example 3}\label{example-3-2}}
\addcontentsline{toc}{paragraph}{Example 3}

\begin{quote}
A solid is obtained by revolving about the y-axis the region bounded by
\(2x=y^2\), \(y=4\), and the \(y\)-axis. Find the volume of the solid.

\textbf{Solution.}

\begin{align*}
V &= \int_0^4 \pi (\frac{y^2}{2})^2 dy \\
&= \int_0^4 \pi (\frac{y^4}{4}) dy \\
&= \frac{\pi y^5}{20} \Bigg|_0^4 \\
V &= \frac{256\pi}{5}
\end{align*}
\end{quote}

\hypertarget{volumes-by-cylindrical-shells}{%
\subsection{Volumes by cylindrical
shells}\label{volumes-by-cylindrical-shells}}

There are times that disks-and-washers technique is not the best way to
solve a volume problem -- e.g.~\(y = 4x-x\) rotated about the
\(y\)-axis.

\hypertarget{volume-of-a-cylindrical-shell}{%
\subsubsection{Volume of a cylindrical
shell}\label{volume-of-a-cylindrical-shell}}

\begin{quote}
Let \(r_1\) be the inner radius of the cylinder, \(r_2\) be the outer
(and larger) radius of the cylinder. \(r\) be the average of both

\begin{align*}
\Delta r &= r_2 - r_1 \\
r &= \frac{r_2+r_1}{2}
\end{align*}

\begin{align*}
V_{\text{cylindrical shell}} &= \pi r_2^2 h - r_1^2 h \\
&= \pi(r_2^2 - r_1^2)h \\
&= \pi(r_2+r_1)h(r_2-r_1) \\
&= 2\pi \Bigg(\frac{r_2+r_1}{2}\Bigg) h \Delta r \\
V_{\text{cylindrical shell}} &= 2\pi r h \Delta r
\end{align*}

Given a curve \(y=f(x)\) in \([a,b]\) rotated about the \(y\)-axis, the
Riemann sum of the volume is

\[ V = \sum 2\pi r h \Delta r \]

In this context, \(r = x\) (horizontal distance), \(h = f(x)\) (vertical
distance), and \(\Delta r = \Delta x\)

\begin{align*}
V &= \lim_{n \to\infty} \sum_{i=1}^n 2\pi x_i^* f(x_i^*) \Delta x \\
&= \int_a^b 2\pi xf(x)dx
\end{align*}
\end{quote}

The volume of a solid obtained by rotating about the \(y\)-axis the
region under the curve \(y = f(x)\) (continuous and nonnegative) from
\(x=a\) (nonnegative) to \(x = b\) is

\[ V = \lim_{n \to\infty} \sum_{i=1}^n 2\pi x_i^* f(x_i^*) \Delta x = \int_a^b 2\pi xf(x)dx \]

\hypertarget{example-1-4}{%
\paragraph*{Example 1}\label{example-1-4}}
\addcontentsline{toc}{paragraph}{Example 1}

\begin{quote}
A solid is obtained by revolving about the \(y\)-axis the region bounded
by \(y = 4x - x^2\) and the \(x\)-axis. Use cylindrical shells to find
the volume of the solid.

\textbf{Solution.}

\begin{align*}
V &= \int_0^4 2 \pi x (4x-x^2) dx \\
&= \int_0^4 2 \pi (4x^2 - x^3)dx \\
&= 2\pi \Big( \frac{4}{3}x^3 - \frac{x^4}{4} \Big) \Bigg|_0^4\\
&= 2\pi \Big[ \frac{4}{3} (4)^3 - \frac{4^4}{4} \Big] - 0\\
&= \frac{128\pi}{3}
\end{align*}
\end{quote}

\hypertarget{example-2-4}{%
\paragraph*{Example 2}\label{example-2-4}}
\addcontentsline{toc}{paragraph}{Example 2}

\begin{quote}
A solid is obatined by revolving about the \(x\)-axis the region bounded
by \(y = \sqrt{x}\), \(y = 2-x\), and the \(x\)-axis. Find the volume of
the solid.

\textbf{Solution.}

\emph{Disks:}

\(y = \sqrt{x}\) and \(y = 2-x\) intersect at \((1,1)\)

\(\forall x \in [0,1]\), \(\sqrt{x} \leq 2-x \implies\) we use
\(\sqrt{x}\) at this interval

\(\forall x \in [1,2]\), \(2-x \leq \sqrt{x} \implies\) we use \(2-x\)
at this interval

\begin{align*}
V &= \int_0^1 \pi (\sqrt{x})^2 dx + \int_1^2 \pi (2-x)^2 dx \\
&= \int_0^1 \pi x dx + \int_1^2 \pi (x^2 - 4x +4) dx
\end{align*}

(This exercise is left to the reader lol don't wanna solve this myself)

\emph{cylindrical shell:}

\(y = \sqrt{x}\) and \(y = 2-x\) intersect at \((1,1)\).

\[ y = \sqrt{x} \implies x = y^2 \] \[ y = 2-x \implies x = 2-y \]

\[ V = \int_0^1 2\pi y (2-y^2-y) dy \]

(Just solve this yourselves)
\end{quote}

\hypertarget{example-3-3}{%
\paragraph*{Example 3}\label{example-3-3}}
\addcontentsline{toc}{paragraph}{Example 3}

\begin{quote}
A solid is obtained by the revolving about the line \(x=-2\) the region
bounded by \(y=x^3\), \(y=8\), and the \(y\)-axis. Find the volume of
the solid.

\textbf{Solution.}

\emph{disks:}
\(\displaystyle V = \int_0^8 \pi \big[ (\sqrt[3]{y} +2)^2 -2^2 \big] dy = \frac{336\pi}{5}\)

\emph{cylindrical shells:}
\(\displaystyle V = \int_0^2 2\pi (x+2)(8-x^3) dx = \frac{336\pi}{5}\)
\end{quote}

\begin{center}\rule{0.5\linewidth}{0.5pt}\end{center}

\hypertarget{techniques-of-integration}{%
\section{Techniques of integration}\label{techniques-of-integration}}

\hypertarget{integration-by-parts}{%
\subsection{Integration by parts}\label{integration-by-parts}}

\hypertarget{product-rule-and-the-differentials}{%
\subsubsection{Product rule and the
differentials}\label{product-rule-and-the-differentials}}

\begin{quote}
Recalling the product rule in derivatives,

\begin{align*}
\frac{d}{dx} (f(x)g(x)) &= f(x)g'(x) + g(x)f'(x)\\
d(f(x)g(x)) &= f(x)g'(x)dx + g(x)f'(x)dx
\end{align*}

Let \(u = f(x)\), \(v=g(x)\)

\(du = f'(x)dx\), \(dv = g'(x)dx\)

\begin{align*}
d(f(x)g(x)) = f(x)g'(x)dx +g(x)f'(x)dx \implies d(uv) &= udv + vdu\\
udv &= d(uv) - vdu\\
\int udv &= \int d(uv)- \int vdu\\
\int udv &= uv - \int vdu
\end{align*}
\end{quote}

\[ \int f(x)g'(x)dx = f(x)g(x) - \int g(x)f'(x)dx \] Letting
\(u = f(x)\), \(v=g(x) \implies du = f'(x)dx\), \(dv = g'(x)dx\),

\[ \int udv = uv - \int vdu \]

\begin{quote}
This is also called the \textbf{integration-by-parts formula}.
\end{quote}

\hypertarget{integration-by-parts-and-definite-integrals}{%
\subsubsection{Integration by parts and definite
integrals}\label{integration-by-parts-and-definite-integrals}}

Combining the integration-by-parts formula and FTC2,

\[ \int_a^b f(x)g'(x)dx = f(x)g(x) \Big|_a^b - \int_a^b g(x)f'(x)dx \]

\hypertarget{example-1-5}{%
\paragraph*{Example 1}\label{example-1-5}}
\addcontentsline{toc}{paragraph}{Example 1}

\begin{quote}
Evaluate \(\displaystyle\int \ln x dx\)

\textbf{Preliminary work.}

\[ \int \ln x dx \implies x > 0 \]

Let \(u = 1\), \(dv = \ln x dx\)

\(du = 0\), \(v = \int \ln x dx\)

Note that this solution is not correct because we just went back to our
original statement, leading us nowhere.

\textbf{Solution.} Let \(u = \ln x\), \(dv = dx\)

\(du = \frac{1}{x}dx\), \(v = \int dx = x + C\)

Following the integration-by-parts formula,

\begin{align*}
\int \ln x dx &= (\ln x)(x+c) - \int (x+C)\frac{1}{x} dx\\
&= x \ln x + C \ln x - \int (1+\frac{C}{x})dx\\
&= x \ln x + C \ln x - (x + C \ln |x| + C_1)\\
&= x \ln x + C \ln x - x - C \ln |x| - C_1\text{;  note that  }x > 0\text{, thus }\ln|x| = \ln x\\
&= x \ln x - x + C
\end{align*}
\end{quote}

\begin{quote}
Note that the contribution of \(+C\) in \(v\) just cancels at
\(uv - \int vdu\), so it is \textbf{not necessary} to put \(+C\) when
using integration by parts.
\end{quote}

\hypertarget{example-2-5}{%
\paragraph*{Example 2}\label{example-2-5}}
\addcontentsline{toc}{paragraph}{Example 2}

\begin{quote}
Evaluate \(\displaystyle \int t^2 \sin {\beta t} dt\)

\textbf{Solution.} Let \(u = t^2\), \(dv = \sin {\beta t} dt\)

\(du = 2tdt\),
\(v = \int \sin {\beta t} dt = -\dfrac{\cos {\beta t}}{\beta}\)

\begin{align*}
\int t^2 \sin \beta t dt  = -\frac{t^2\cos{\beta t}}{\beta} - \int -\frac{2t\cos{\beta t}}{\beta}dt
= -\frac{t^2\cos{\beta t}}{\beta} - \int \frac{2}{\beta} t \cos{\beta t}dt
\end{align*}

Let \(u = t\), \(dv = \cos{\beta t}dt\)

\(du = dt\), \(v = \dfrac{\sin{\beta t}}{{\beta}}\)

\begin{align*}
-\frac{t^2\cos{\beta t}}{\beta} - \int \frac{2}{\beta} t\cos{\beta t}dt &= -\frac{t^2\cos{\beta t}}{\beta} - \frac{2}{\beta}\Big(\frac{t \sin{\beta t}}{{\beta}} - \int \frac{\sin{\beta t}}{{\beta}}dt\Big)\\
&= -\frac{t^2\cos{\beta t}}{\beta} - \frac{2t \sin{\beta t}}{\beta^2} - \frac{2}{\beta} \Big(\int \frac{\sin{\beta t}}{\beta} dt\Big)\\
&= -\frac{t^2\cos{\beta t}}{\beta} - \frac{2t \sin{\beta t}}{\beta^2} + \frac{2\cos {\beta t}}{\beta^3} + C
\end{align*}
\end{quote}

\hypertarget{example-3-4}{%
\paragraph*{Example 3}\label{example-3-4}}
\addcontentsline{toc}{paragraph}{Example 3}

\begin{quote}
Evaluate \(\displaystyle \int e^x \sin {\pi x}dx\)

\textbf{Solution.} Let \(u = \sin {\pi x}\), \(dv = e^x dx\)

\(du = \pi \cos {\pi x}dx\), \(v = e^x\)

By integration by parts,

\[ \int e^x \sin {\pi x}dx = e^x \sin {\pi x} - \pi \int e^x \cos {\pi x}dx \]

Let \(u = \cos {\pi x}\), \(dv = e^xdx\)

\(du = -\pi \sin {\pi x}\), \(v = e^x\)

By integration by parts,

\begin{align*}
\int e^x \sin {\pi x}dx = e^x \sin {\pi x} - \pi \int e^x \cos {\pi x}dx &= e^x \sin {\pi x} - \pi \Big(e^x\cos {\pi x} - \pi \int - e^x \sin {\pi x}dx \Big)\\
&=  e^x \sin {\pi x} - \pi e^x\cos {\pi x} - \pi^2 \int  e^x \sin {\pi x}dx \\
\int e^x \sin {\pi x}dx + \pi^2 \int  e^x \sin {\pi x}dx &= e^x \sin {\pi x} - \pi e^x\cos {\pi x}\\
(1 + \pi^2) \int e^x \sin {\pi x}dx &= e^x \sin {\pi x} - \pi e^x\cos {\pi x}\\
\int e^x \sin {\pi x}dx &= \frac{e^x \sin {\pi x} - \pi e^x\cos {\pi x}}{1 + \pi^2} + C
\end{align*}
\end{quote}

\hypertarget{example-4-definite-integral}{%
\paragraph*{Example 4 (definite
integral)}\label{example-4-definite-integral}}
\addcontentsline{toc}{paragraph}{Example 4 (definite integral)}

\begin{quote}
Evaluate \(\displaystyle \int_0^1 \tan^{-1}x dx\).

\textbf{Solution.} We shall first get the antiderivative of the function
before evaluating the definite integral.

Let \(u = \arctan x\), \(dv = dx\)

\(du = \dfrac{1}{x^2+1}dx\) , \(v = x\)

By integration by parts,

\[ \int \tan^{-1}x dx = x\arctan x - \int \frac{x}{x^2+1}dx \]

Let \(u = x^2 + 1 \implies du = 2xdx\)

Using the substitution rule,

\begin{align*}
x\arctan x - \int \frac{x}{x^2+1}dx &= x\arctan x - \frac{1}{2}\int \frac{1}{u}du\\
&= x\arctan x - \frac{\ln |u|}{2}\\
&= x\arctan x - \frac{\ln |x^2+1|}{2}\text{, note that }\forall x \in \mathbb{R}, x^2+1 > 0 \\
&= x\arctan x - \frac{\ln (x^2+1)}{2}
\end{align*}

By FTC2,

\begin{align*}
\int_0^1 \tan^{-1}x dx &= (x\arctan x - \frac{\ln (x^2+1)}{2}) |_0^1\\
&= [\arctan 1 - \frac{\ln (1^2+1)}{2}] - [ - \frac{\ln (0^2+1)}{2}]\\
&= \frac{\pi - 2\ln(2)}{4}
\end{align*}
\end{quote}

\begin{center}\rule{0.5\linewidth}{0.5pt}\end{center}

\hypertarget{trigonometric-integrals}{%
\subsection{Trigonometric integrals}\label{trigonometric-integrals}}

\hypertarget{trigonometric-identities}{%
\subsubsection{Trigonometric
identities}\label{trigonometric-identities}}

\begin{enumerate}
\def\labelenumi{\arabic{enumi}.}
\item
  \(\displaystyle \sin^2 x + \cos^2 x = 1\)
\item
  \(\displaystyle \tan^2 x + 1 = sec^2 x\)
\item
  \(\displaystyle \cot^2 x + 1 = csc^2 x\)
\item
  \(\displaystyle \sin^2 x = \frac{1}{2}(1-\cos 2x)\)
\item
  \(\displaystyle \cos^2 x = \frac{1}{2}(1+\cos 2x)\)
\item
  \(\displaystyle \sin A\cos B = \frac{1}{2}[\sin(A-B)+\sin(A+B)]\)
\item
  \(\displaystyle \sin A\sin B = \frac{1}{2}[\cos(A-B) ]- \cos(A+B)]\)
\item
  \(\displaystyle \cos A\cos B = \frac{1}{2}[\cos(A-B)+\cos(A+B)]\)
\end{enumerate}

\begin{center}\rule{0.5\linewidth}{0.5pt}\end{center}

\hypertarget{integrals-of-trigonometric-functions}{%
\subsubsection{Integrals of trigonometric
functions}\label{integrals-of-trigonometric-functions}}

\begin{enumerate}
\def\labelenumi{\arabic{enumi}.}
\tightlist
\item
  \(\displaystyle \int \tan x dx = \ln |\sec x| +C\)
\end{enumerate}

\begin{quote}
\begin{align*}
\int \tan x dx = \int \frac{\sin x}{\cos x} dx &= -\ln|\cos x| + C\\
&= \ln|\sec x| + C
\end{align*}
\end{quote}

\begin{enumerate}
\def\labelenumi{\arabic{enumi}.}
\setcounter{enumi}{1}
\tightlist
\item
  \(\displaystyle \int \sec x dx = \ln|\sec x + \tan x| + C\)
\end{enumerate}

\begin{quote}
\begin{align*}
\int \sec x dx &= \int sec x \frac{\sec x + \tan x}{\sec x + \tan x} dx\\
&= \int \frac{\sec^2 + \sec x\tan x}{\sec x + \tan x}dx\\
&= \ln |\sec x + \tan x| + C
\end{align*}
\end{quote}

\begin{enumerate}
\def\labelenumi{\arabic{enumi}.}
\setcounter{enumi}{2}
\item
  \(\displaystyle \int \cot x dx = \ln|\sin x| + C\)
\item
  \(\displaystyle \int \csc x dx = \ln|\csc x - \cot x| + C\)
\end{enumerate}

\begin{center}\rule{0.5\linewidth}{0.5pt}\end{center}

\hypertarget{example-1-6}{%
\paragraph*{Example 1}\label{example-1-6}}
\addcontentsline{toc}{paragraph}{Example 1}

\begin{quote}
Evaluate \(\displaystyle \int \sqrt{cos \theta} sin^3 \theta d\theta\).

\textbf{Solution.}

\begin{align*}
\int \sqrt{\cos \theta} \sin^3 \theta d\theta &= \int \sqrt{\cos \theta}\sin^2 \theta \sin \theta d\theta \\
&= \int \sqrt{\cos \theta}(1 - \cos^2 \theta) \sin \theta d\theta
\end{align*}

Let \(u = \sqrt{\cos \theta}\)

\(\implies u^2 = \cos \theta\)

\(\implies 2udu = -\sin \theta d\theta\)

\(\implies -2udu = \sin \theta d\theta\)

\begin{align*}
\int \sqrt{\cos \theta}(1 - \cos^2 \theta) \sin \theta d\theta &= \int u(1-u^4)(-2u)du \\
&= \int (-2u^2 + 2u^6)du \\
&= -\frac{2}{3}u^3 +\frac{2}{7} u^7 + C \\
&= -\frac{2}{3}(\cos \theta)^{3/2} + \frac{2}{7}(\cos \theta)^{7/2}+C
\end{align*}
\end{quote}

\hypertarget{example-2-6}{%
\paragraph*{Example 2}\label{example-2-6}}
\addcontentsline{toc}{paragraph}{Example 2}

\begin{quote}
Evaluate \(\displaystyle \int \tan^2 t \sec^4 t dt\).

\textbf{Solution.}

\begin{align*}
\int \tan^2 t \sec^4 t dt &= \int \tan^2 t \sec^2 t \sec^2 t dt \\
&= \int \tan^2 t (1 + \tan^2 t) \sec^2 t dt
\end{align*}

Let \(u = \tan t \implies du = \sec^2 t dt\).

\begin{align*}
\int \tan^2 t (1 + \tan^2 t) \sec^2 t dt &= \int u^2 (1+u^2) du \\
&= \int (u^2 + u^4)du\\
&= \frac{u^3}{3} + \frac{u^5}{5}\\
&= \frac{\tan^5 t}{5} + \frac{\tan^3 t}{3} + C
\end{align*}
\end{quote}

\hypertarget{example-3-definite-integral}{%
\paragraph*{Example 3 (definite
integral)}\label{example-3-definite-integral}}
\addcontentsline{toc}{paragraph}{Example 3 (definite integral)}

\begin{quote}
Evaluate \(\displaystyle \int_{\pi/4}^{\pi/2} \csc^5 x \cot^3 x dx\).

\begin{align*}
\int_{\pi/4}^{\pi/2} \csc^5 x \cot^3 x dx &= \int_{\pi/4}^{\pi/2} \csc^4 x \cot^2 x \csc x\cot x dx\\
&= \int_{\pi/4}^{\pi/2} \csc^4 x (\csc^2 x - 1) \csc x\cot x dx
\end{align*}

Let
\(u = \csc x \implies du = -\csc x \cot x dx \implies -du = \csc x \cot x dx\)

\(x = \dfrac{\pi}{4} \implies u = \sqrt{2}\)

\(x = \dfrac{\pi}{2} \implies u = 1\)

\begin{align*}
\int_{\pi/4}^{\pi/2} \csc^4 x (\csc^2 x - 1) \csc x\cot x dx &= \int_{\sqrt{2}}^1 u^4(u^2-1)(-du)\\
&= \int_{\sqrt{2}}^1 - u^6 + u^4 du\\
&= \int_1^{\sqrt{2}} u^6 - u^4 du\\
&= \frac{u^7}{7} - \frac{u^5}{5} \Bigg|_1^{\sqrt{2}}\\
&= \frac{8\sqrt{2}}{7} - \frac{4\sqrt{2}}{5} - \frac{1}{7} + \frac{1}{5}\\
&= \frac{40\sqrt{2}-28\sqrt{2} -5 +7}{35}\\
&= \frac{12\sqrt{2} +2}{35}
\end{align*}
\end{quote}

\hypertarget{example-4-definite-integral-1}{%
\paragraph*{Example 4 (definite
integral)}\label{example-4-definite-integral-1}}
\addcontentsline{toc}{paragraph}{Example 4 (definite integral)}

\begin{quote}
Evaluate \(\displaystyle \int_0^{\pi/2} \cos 5t \cos 10t dt\).

\begin{align*}
\int_0^{\pi/2} \cos 5t \cos 10t dt &= \int_0^{\pi/2} \frac{1}{2} [\cos(5t-10t) + cos(5t+10t)]dt\\
&= \int_0^{\pi/2} \frac{1}{2} [\cos(-5t)+\cos 15t]dt \text{; note that } \cos(-5t) = \cos{5t} \;\forall t\\
&= \int_0^{\pi/2} \frac{1}{2} \cos 5t + \frac{1}{2}\cos 15t\\
&= \frac{1}{10}\sin 5t + \frac{1}{30} \sin 15t \Bigg|_0^{\pi/2}\\
&= \frac{1}{10}\sin {\frac{5\pi}{2}} + \frac{1}{30} \sin {\frac{15\pi}{2}} - 0\\
&= \frac{1}{15}
\end{align*}
\end{quote}

\hypertarget{example-5}{%
\paragraph*{Example 5}\label{example-5}}
\addcontentsline{toc}{paragraph}{Example 5}

\begin{quote}
Evaluate \(\displaystyle \int \frac{\sin^2(1/t)}{t^2}dt\).

Let
\(u = \dfrac{1}{t} \implies du = -\dfrac{1}{t^2} \implies -du = \dfrac{1}{t^2}\)

\[ \int \frac{\sin^2(1/t)}{t^2}dt = \int -\sin^2 u du \]

Note that \(\sin^2 u = \frac{1}{2}(1-\cos 2u)\)

\begin{align*}
\int -\sin^2 u du &= \int -\frac{1}{2}(1-cos2u)du\\
&= -\frac{1}{2}(u - \frac{sin 2u}{2})\\
&= -\frac{1}{2t} + \frac{\sin (2/t)}{4} + C
\end{align*}
\end{quote}

\begin{center}\rule{0.5\linewidth}{0.5pt}\end{center}

\hypertarget{trigonometric-substitution}{%
\subsection{Trigonometric
substitution}\label{trigonometric-substitution}}

\hypertarget{circular-functions}{%
\subsubsection{Circular functions}\label{circular-functions}}

\[ x^2 + y^2 = r^2 \]

which proves that trigonometric functions are also circular functions

\[\sin \theta = \frac{y}{r} \implies y = r\sin \theta\]
\[\cos \theta = \frac{x}{r} \implies x = r\cos \theta\]

We can therefore rename \((x,y)\) to \((r\cos \theta, r\sin \theta)\).
It implies that for every \((x,y)\) coordinates, there corresponds
\((r, \theta)\) coordinates.

\hypertarget{recall}{%
\paragraph*{Recall}\label{recall}}
\addcontentsline{toc}{paragraph}{Recall}

\begin{quote}
Evaluate \(\displaystyle \int_{-2}^{2} \sqrt{4-x^2}dx\).

\textbf{Intuitive solution.} Observe that the function is a semicircle
with r=2, thus the integral looks for the area of the semicircle.

\[ \int_{-2}^{2} \sqrt{4-x^2}dx = \frac{1}{2}\pi2^2 = 2\pi\]

\textbf{Solution.} Let \(y = \sqrt{4-x^2}\).

\(y^2 = 4-x^2\)

\(x^2 + y^2 = 4\)

\(r^2 = 4 \implies r = 2\)

\((x,y) = (x,\sqrt{4-x^2}) = (2\cos \theta, 2\sin \theta)\)

\(x = 2 \cos \theta \implies dx = 2(-\sin \theta)d\theta\)

\(\sqrt{4-x^2} = 2 \sin \theta\)

\begin{align*}
\int_{-2}^{2} \sqrt{4-x^2}dx &= \int_{\pi}^0(2\sin \theta)(-2\sin\theta d\theta) \text{; note that we converted the } x \text{-value to} \theta \text{-value}\\
&= \int_{\pi}^0 -4\sin^2 \theta d\theta\\
&= \int_0^{\pi} -4\sin^2 \theta d\theta\\
&= \int_0^{\pi} \frac{4}{2}(1-cos 2\theta) d\theta\\
&= \int_0^{\pi} 2(1-cos 2\theta) d\theta\\
&= 2x - \frac{sin 2\theta}{2} \Bigg|_0^{\pi}\\
&= 2\pi - 0 - 0\\
&= 2\pi
\end{align*}
\end{quote}

\hypertarget{trigonometric-substitutions}{%
\subsubsection{Trigonometric
substitutions}\label{trigonometric-substitutions}}

\begin{longtable}[]{@{}ccc@{}}
\toprule
\begin{minipage}[b]{0.30\columnwidth}\centering
Expression\strut
\end{minipage} & \begin{minipage}[b]{0.30\columnwidth}\centering
Substitution\strut
\end{minipage} & \begin{minipage}[b]{0.30\columnwidth}\centering
Identity\strut
\end{minipage}\tabularnewline
\midrule
\endhead
\begin{minipage}[t]{0.30\columnwidth}\centering
\(\sqrt{a^2-x^2}\)\strut
\end{minipage} & \begin{minipage}[t]{0.30\columnwidth}\centering
\(x = a\sin \theta, -\dfrac{\pi}{2} \leq \theta \leq \dfrac{\pi}{2}\)\strut
\end{minipage} & \begin{minipage}[t]{0.30\columnwidth}\centering
\(1 - \sin^2 \theta = \cos^2 \theta\)\strut
\end{minipage}\tabularnewline
\begin{minipage}[t]{0.30\columnwidth}\centering
\strut
\end{minipage} & \begin{minipage}[t]{0.30\columnwidth}\centering
\strut
\end{minipage} & \begin{minipage}[t]{0.30\columnwidth}\centering
\strut
\end{minipage}\tabularnewline
\begin{minipage}[t]{0.30\columnwidth}\centering
\(\sqrt{a^2+x^2}\)\strut
\end{minipage} & \begin{minipage}[t]{0.30\columnwidth}\centering
\(x = a\tan \theta, -\dfrac{\pi}{2} \leq \theta \leq \dfrac{\pi}{2}\)\strut
\end{minipage} & \begin{minipage}[t]{0.30\columnwidth}\centering
\(1 + \tan^2 \theta = \sec^2 \theta\)\strut
\end{minipage}\tabularnewline
\begin{minipage}[t]{0.30\columnwidth}\centering
\strut
\end{minipage} & \begin{minipage}[t]{0.30\columnwidth}\centering
\strut
\end{minipage} & \begin{minipage}[t]{0.30\columnwidth}\centering
\strut
\end{minipage}\tabularnewline
\begin{minipage}[t]{0.30\columnwidth}\centering
\(\sqrt{x^2-a^2}\)\strut
\end{minipage} & \begin{minipage}[t]{0.30\columnwidth}\centering
\(x = a \sec \theta, 0 \leq \theta \leq \dfrac{\pi}{2}\) or
\(\pi \leq \theta \leq \dfrac{3\pi}{2}\)\strut
\end{minipage} & \begin{minipage}[t]{0.30\columnwidth}\centering
\(\sec^2 \theta - 1 = \tan^2 \theta\)\strut
\end{minipage}\tabularnewline
\begin{minipage}[t]{0.30\columnwidth}\centering
\strut
\end{minipage} & \begin{minipage}[t]{0.30\columnwidth}\centering
\strut
\end{minipage} & \begin{minipage}[t]{0.30\columnwidth}\centering
\strut
\end{minipage}\tabularnewline
\bottomrule
\end{longtable}

\hypertarget{example-1-7}{%
\paragraph*{Example 1}\label{example-1-7}}
\addcontentsline{toc}{paragraph}{Example 1}

\begin{quote}
Evaluate \(\displaystyle \int_0^1 \frac{dx}{(x^2+1)^2}\)

\[ \int_0^1 \frac{dx}{(x^2+1)^2} = \int_0^1 \frac{dx}{(\sqrt{x^2+1})^4} \]

Visualizing a right triangle with legs \(x\) and \(1\),

\(x = \tan \theta \implies dx = \sec^2 \theta d\theta\)

\(\cos \theta =\dfrac{1}{\sqrt{x^2 +1}} \implies \sqrt{x^2 +1} = \sec \theta\)

\(x = 0 \implies 0 = \tan \theta \implies \theta = 0\)

\(x = 1 \implies \theta = \dfrac{\pi}{4}\)

\begin{align*}
\int_0^1 \frac{dx}{(\sqrt{x^2+1})^4} &= \int_0^{\pi/4} \frac{\sec^2 \theta}{\sec^4 \theta} d\theta \\
&= \int_0^{\pi/4} \frac{1}{\sec^2 \theta} d\theta \\
&= \int_0^{\pi/4} \cos^2 \theta d\theta \\
&= \int_0^{\pi/4} \frac{1}{2} (1 + \cos 2 \theta) d\theta  \\
&= \frac{1}{2} \Big(\theta + \frac{\sin 2\theta}{2}\Big) \Bigg|_0^{\pi/4}\\
\end{align*}

(solution to be continued)
\end{quote}

\hypertarget{example-2-7}{%
\paragraph*{Example 2}\label{example-2-7}}
\addcontentsline{toc}{paragraph}{Example 2}

\begin{quote}
Evaluate \(\displaystyle \int \frac{\sqrt{x^2-1}}{x^4}dx\).

Solution.

By trigonometric substitution,

\(x = \sec \theta \implies dx = \sec \theta \tan \theta d\theta\)

\(\sqrt{x^2 -1} = \tan \theta\)

\begin{align*}
\int \frac{\sqrt{x^2-1}}{x^4}dx &= \int \frac{\tan\theta}{\sec^4 \theta} (\sec \theta \tan \theta d\theta)\\
&= \int \frac{\tan^2\theta}{\sec^3 \theta} d\theta\\
&= \int \sin^2\theta \cos\theta d\theta\\
&= \frac{\sin^3 \theta}{3} + C\\
&= \frac{1}{3}(\frac{\sqrt{x^2-1}}{x})^3 + C\\
&= \frac{(x^2-1)^{3/2}}{3x^3} + C\\
\end{align*}
\end{quote}

\hypertarget{example-3-5}{%
\paragraph*{Example 3}\label{example-3-5}}
\addcontentsline{toc}{paragraph}{Example 3}

\begin{quote}
Evaluate \(\displaystyle \int \frac{x^2}{(3+4x-4x^2)^{3/2}}dx\).

\textbf{Solution.}

\[ \int \frac{x^2}{(3+4x-4x^2)^{3/2}}dx = \int \frac{x^2}{[4 - (2x-1)^2]^{3/2}} dx \]

Let there be a triangle with legs \(2x-1\) and \(\sqrt{4-(2x-1)^2}\)

\(2x-1 = 2 \sin \theta \implies x = \dfrac{1+2\sin\theta}{2}\)

\(2dx = 2\cos \theta d \theta\)

\(dx = \cos \theta d\theta\)

\(\sqrt{4-(2x-1)^2} = 2\cos \theta\)

\begin{align*}
\int \frac{x^2}{[4 - (2x-1)^2]^{3/2}} dx &= \int \frac{\Big(\dfrac{1+ 2\sin\theta}{2}\Big)^2}{2\cos^3 \theta} \cos\theta d\theta\\
&= \int \frac{1 + 4\sin \theta + 4 \sin^2 \theta}{32 \cos \theta} d\theta\\
&= \frac{1}{32} \int(\sec^2\theta + 4 \tan \theta \sec\theta + 4 \tan^2 \theta) d\theta\\
&= \frac{1}{32}\tan \theta + \frac{1}{8} \sec \theta + \frac{1}{8} \tan \theta - \frac{1}{8} \theta + C\\
&= \frac{5}{32}\tan \theta + \frac{1}{8} \sec \theta - \frac{1}{8} \theta + C\\ 
&= \frac{5}{32} \frac{2x-1}{\sqrt{3+4x-4x^2}} + \frac{1}{8} \frac{2}{\sqrt{3+4x-4x^2}} - \frac{1}{8}\arcsin \Big(\frac{2x-1}{2}\Big) + C
\end{align*}
\end{quote}

\hypertarget{integration-of-rational-functions-by-partial-fractions}{%
\subsection{Integration of rational functions by partial
fractions}\label{integration-of-rational-functions-by-partial-fractions}}

\hypertarget{quick-review}{%
\paragraph*{Quick review}\label{quick-review}}
\addcontentsline{toc}{paragraph}{Quick review}

\begin{quote}
\(\displaystyle\int \frac{2}{3-2x}dx = \ln |3-2x| + C\)

\(\displaystyle\int \frac{2}{(3-2x)^2} dx = \frac{1}{3-2x} +C\)

\(\displaystyle\int \frac{x}{x^2 + 4} dx = \frac{1}{2}\ln(x^2+4) + C \text{; note that } x^2 + 4 \geq 0 \;\forall x \in \mathbb{R}\)

\(\displaystyle\int \frac{x}{(x^2 +4)^2} dx = -\frac{1}{2(x^2+4)} + C\)

\(\displaystyle\int \frac{1}{x^2 + a^2} dx = \frac{1}{a}\arctan \left( \frac{x}{a} \right) + C\)
\end{quote}

\hypertarget{partial-fractions}{%
\subsubsection{Partial fractions}\label{partial-fractions}}

\hypertarget{example-4}{%
\paragraph*{Example}\label{example-4}}
\addcontentsline{toc}{paragraph}{Example}

\begin{quote}
\[ \frac{1}{x+2} - \frac{1}{x-3} = \frac{x-2-(x+2)}{(x+2)(x-3)} = \frac{-5}{(x+2)(x-3)} \]

This is an identity because it is true for all \(x\).

\[\implies \frac{-5}{(x+2)(x-3)} = \frac{A}{x+2} + \frac{B}{x-3},\; A,B \in \mathbb{R} \]

\begin{align*} 
(x+2)(x-3) \Bigg[ \frac{-5}{(x+2)(x-3)} &= \frac{A}{x+2} + \frac{B}{x-3} \Bigg] (x+2)(x-3)\\
-5 &= A(x-3) + B(x+2)\\
-5 &= Ax -3A + Bx +2B\\
-5 &= (A+B)x + (-3A + 2B)
\end{align*}

By systems of equations,

\[
\begin{cases}
A+B = 0 \implies A = -B\\
-3A + 2B = -5 \implies = -3(-B) + 2B = -5 \implies B = -1, A = 1
\end{cases}
\]

By substitution,

\begin{align*}
x = 3: \;-5 &= 0 + B(3+2)\\
B &= -1
\end{align*}

\begin{align*}
x= -2: \; -5 &= A(-2-3) + 0\\
A &= 1
\end{align*}
\end{quote}

\hypertarget{applications-ii}{%
\section{Applications II}\label{applications-ii}}

\hypertarget{arc-length}{%
\subsection{Arc length}\label{arc-length}}

\begin{quote}
Suppose a continuous function \(f(x)\) continuous on \([a,b]\). We want
to get the length of the curve from \([a,b]\).

Let \(\Delta x\) be the spacing of the values of \(x\) when divided in
to \(n\) equal intervals.

\[\Delta x = x_i - x_{i-1} = \frac{b-a}{n} \]

We have two adjacent points, \(P_{i-1}(x, f(x))\) and
\(P_i(x+\Delta x, f(x + \Delta x))\). Rewriting the coordinates, we get
\(P_{i-1}(x_{i-1}, f(x_{i-1}))\) and \(P_i(x_i, f(x_i))\). Getting the
distance between these two points,

\[|P_{i-1}P_i| = \sqrt{(x_i-x_{i-1})^2 + \left[ f(x_i)-f(x_{i-1}) \right]^2}\]

Recalling MVT for derivatives,

\begin{itemize}
\item
  \(f\) is continuous on \([x_{i-1},x_i]\)
\item
  \(f\) is differentiable on \((x_{i-1},x_i)\)
\end{itemize}

\(\implies \exists x_i^* \in (x_{i-1},x_i)\) such that

\[f'(x_i^*) = \frac{f(x_i) - f(x_{i-1})}{x_i - x_{i-1}}\]

Going back to \(|P_{i-1}P_i|\),

\begin{align*}
|P_{i-1}P_i| &= \sqrt{(x_i-x_{i-1})^2 + \left[ f(x_i)-f(x_{i-1}) \right]^2}\\
&= \sqrt{(x_i-x_{i-1})^2 + \left[ f'(x_i^*) (x_i - x_{i-1}) \right]^2}\\
&= \sqrt{(\Delta x)^2 + \left[ f'(x_i^*) \Delta x \right]^2}\\
&= \sqrt{1+[f'(x_i^*)]^2}\Delta x
\end{align*}

Let the length of the curve be \(L\).

\begin{align*}
L &= \lim_{n\to\infty} \sum_{i=1}^n |P_{i-1}P_i|\\
&= \lim_{n\to\infty} \sum_{i=1}^n \sqrt{1+[f'(x_i^*)]^2}\Delta x\\
L &= \int_a^b \sqrt{1+[f'(x)]^2}dx\\
\end{align*}
\end{quote}

\hypertarget{the-arc-length-formula}{%
\subsubsection{The arc length formula}\label{the-arc-length-formula}}

If \(f'\) is continuous on \([a,b]\), then the length \(L\) of the curve
\(y = f(x)\), \(a \leq x \leq b\), is given by

\[ L = \int_a^b \sqrt{1+[f(x)]^2}dx \]

\begin{center}\rule{0.5\linewidth}{0.5pt}\end{center}

\hypertarget{differential-equations}{%
\subsection{Differential equations}\label{differential-equations}}

\begin{quote}
incomplete
\end{quote}

\hypertarget{models-for-population-growth}{%
\subsubsection{Models for population
growth}\label{models-for-population-growth}}

Malthus (c.~1798) said that in an ideal population, the rate of growth
of a population is proportional to the size of the population.

Let \(P\) be a function of time stating the number of individuals in a
population at time \(t\).

\begin{align*}
\frac{dP}{dt} &\propto P\\
\implies\frac{dP}{dt} &= kP; k \text{ is the proportionality constant}\\
\frac{\dfrac{dP}{dt}}{P} &= k
\end{align*}

This equations is known as the \textbf{simple growth model}.

Verholst (c.~1838) mentioned that Malthus's population model is only
correct for small populations, and that population growth is bounded by
environmental restrictions.

Let \(M\) be the carrying capacity of the environment.

Then when the population surpasses \(M\), the population growth pattern
changes.

\[\frac{\dfrac{dP}{dt}}{P} = k\left(1-\frac{P}{M}\right)\]

Note that when \(P > M\), then the proportion becomes negative.

This is knows as the \textbf{logistic model}.

\hypertarget{differential-equation}{%
\subsubsection{Differential equation}\label{differential-equation}}

A \textbf{differential equation} is an equation that contains an unknown
function and one or more of its derivatives.

\hypertarget{differential-equation-terminologies}{%
\paragraph*{Differential equation
terminologies}\label{differential-equation-terminologies}}
\addcontentsline{toc}{paragraph}{Differential equation terminologies}

\begin{itemize}
\item
  \textbf{order} - highest order of the derivative that appears
\item
  \textbf{solution} - the equation that satisfies the differential
  equation
\item
  \textbf{solve} - refers to finding \textbf{ALL} equations that
  satisfies the differential equation
\item
  \textbf{initial-value problem} - finding the solution to the
  differential equation with an initial condition given -- e.g.~a
  coordinate

  \hypertarget{simple-growth-model}{%
  \subsubsection{Simple growth model}\label{simple-growth-model}}

  The solution of the initial-value problem

  \[ \frac{dP}{dt} = kP, P(0) = P_0 \]

  is

  \[P(t) = P_0e^{kt} \]

  \hypertarget{logistic-model}{%
  \subsubsection{Logistic model}\label{logistic-model}}

  The solution fo the initial-value problem

  \[\dfrac{\dfrac{dP}{dt}}{P} = k \left(1-\frac{P}{M}\right)\]

  is

  \[ P(t) = \frac{M}{1-Ae^{kt}}, A = \frac{M-P_0}{P_0} \]
\end{itemize}

\end{document}
