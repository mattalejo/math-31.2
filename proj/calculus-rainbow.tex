% Preamble
\documentclass[a4paper,12pt]{article}
\usepackage[utf8]{inputenc}
\usepackage{amsmath}
\usepackage{amssymb}
\usepackage{amsfonts}
\usepackage{tikz}
\usepackage{gensymb}
\usepackage[margin=1in]{geometry}
\usepackage{setspace}
\usepackage{float}
\usepackage{xcolor}
\usepackage[colorlinks=true,linkcolor=black,anchorcolor=black,citecolor=black,filecolor=black,menucolor=black,runcolor=black,urlcolor=black]{hyperref}
\usepackage{caption}
\usepackage{subcaption}

\usepackage{graphicx,tipa}

\usepackage[style=numeric, sorting=none]{biblatex}
\addbibresource{calculus-rainbow.bib}


\usepackage{pgfplots}
\pgfplotsset{compat=1.15}
\usepackage{mathrsfs}
\usetikzlibrary{arrows}
\pagestyle{empty}
\newcommand{\reflection}
{
\begin{tikzpicture}[line cap=round,line join=round,>=triangle 45,x=1cm,y=1cm]
\clip(-4.100471663842028,-0.3) rectangle (3.290690084294631,4.2126654282396805);
\draw [shift={(0,0)},line width=0.5pt,color=qqwuqq,fill=qqwuqq,fill opacity=0.10000000149011612] (0,0) -- (90:0.1986871437671145) arc (90:135:0.1986871437671145) -- cycle;
\draw [shift={(0,0)},line width=0.5pt,color=qqwuqq,fill=qqwuqq,fill opacity=0.10000000149011612] (0,0) -- (45:0.1986871437671145) arc (45:90:0.1986871437671145) -- cycle;
\draw [line width=0.5pt,color=qqzzff,domain=-4.100471663842028:3.290690084294631] plot(\x,{(-0-0*\x)/1});
\draw [line width=0.5pt,dash pattern=on 1pt off 1pt] (0,-1.9135215045796783) -- (0,4.2126654282396805);
\draw [line width=0.5pt,color=ffqqqq,domain=-4.100471663842028:0] plot(\x,{(-0--1.7073841328406107*\x)/-1.7073841328406107});
\draw [line width=0.5pt,color=ffqqqq,domain=0:3.290690084294631] plot(\x,{(-0--1.345*\x)/1.345});
\draw[line width=0.5pt,color=ttffcc,fill=ttffcc,fill opacity=0.25](-4.1667007117644,0.0004979803768565896)--(-4.1667007117644,-1.9135215045796783)--(3.356919132217002,-1.9135215045796783)--(3.356919132217002,0.0004979803768565896);
\draw [line width=0.5pt] (-1.991118522263103,1.991118522263103)-- (-2,0);
\draw [line width=0.5pt] (0,0)-- (1,0);
\draw [line width=0.5pt] (0,0)-- (-2,0);
\draw [line width=0.5pt] (1,0)-- (1,1);
\draw [line width=0.5pt] (-2,0)-- (1,0);
\begin{scriptsize}
\draw[color=qqzzff] (-4.040865520711893,0.12964462382548092) node {};
\draw [fill=uuuuuu] (0,0) circle (1pt);
\draw[color=uuuuuu] (0.05208964089066468,0.12964462382548092) node [below right = 1pt]{$P$};
\draw[color=ffqqqq] (-0.908231553983722,0.8581641509715668) node [above right]{$d_1$};
\draw[color=ffqqqq] (0.7673633584522769,0.6793457215811639) node [above left]{$d_2$};
\draw [fill=uuuuuu] (1,1) circle (1pt);
\draw[color=uuuuuu] (1.0521482645184743,1.1297032474532898) node [above]{$B$};
\draw [fill=black] (-1.991118522263103,1.991118522263103) circle (1pt);
\draw[color=black] (-1.9414047015727174,2.136384775873336) node {$A$};
\draw[color=black] (-2.0804857022096974,1.0767200091153926) node [left]{$y_1$};
\draw[color=black] (0.5223158811395023,-0.02930509118821055) node [above]{$a-x$};
\draw[color=black] (-0.9744606019060935,0.18262786216337806) node {$x$};
\draw[color=black] (1.1316231220253201,0.5800021496976068) node [right]{$y_2$};
\draw[color=black] (-0.4777427424883073,-0.02930509118821055) node [below]{$a$};
\draw[color=qqwuqq] (0.2507767846577792,0.1561362429944295) node [above left=4pt]{$\theta_1$};
\draw[color=qqwuqq] (0.31038292778791354,0.1561362429944295) node [above = 4pt]{$\theta_2$};
\end{scriptsize}
\end{tikzpicture}
}

\newcommand{\snells}
{
\begin{tikzpicture}[line cap=round,line join=round,>=triangle 45,x=1cm,y=1cm]
\clip(-3.3011195111302225,-3) rectangle (3.004591388226776,3.962995645536179);
\draw [shift={(0,0)},line width=0.5pt,color=qqwuqq,fill=qqwuqq,fill opacity=0.10000000149011612] (0,0) -- (90:0.27336896962530344) arc (90:120.96375653207352:0.27336896962530344) -- cycle;
\draw [shift={(0,0)},line width=0.5pt,color=qqwuqq,fill=qqwuqq,fill opacity=0.10000000149011612] (0,0) -- (-90:0.27336896962530344) arc (-90:-45:0.27336896962530344) -- cycle;
\draw [line width=0.5pt,color=qqzzcc,domain=-3.3011195111302225:3.004591388226776] plot(\x,{(-0-0*\x)/1});
\draw [line width=0.5pt,dash pattern=on 1pt off 1pt] (0,-3.4658809179106473) -- (0,4.962995645536179);
\draw [line width=0.5pt,color=ffqqqq,domain=-3.3011195111302225:0] plot(\x,{(-0--0.7641661333767144*\x)/-0.45849968002602864});
\draw [line width=0.5pt,color=ffqqqq,domain=0:3.004591388226776] plot(\x,{(-0-0.6418678951944565*\x)/0.6418678951944565});
\draw [line width=0.5pt] (-1.751712643960778,0)-- (0,0);
\draw [line width=0.5pt] (0,0)-- (1.28,0);
\draw [line width=0.5pt] (-1.751712643960778,0)-- (-1.75,2.9166666666666665);
\draw [line width=0.5pt] (1.28,0)-- (1.28,-1.28);
\draw[line width=0.5pt,color=ttffcc,fill=ttffcc,fill opacity=0.25](-3.3922425010053234,-0.003207302656816285)--(-3.3922425010053234,-3.4658809179106473)--(3.0957143781018774,-3.4658809179106473)--(3.0957143781018774,-0.003207302656816285);
\begin{scriptsize}
\draw[color=qqzzcc] (-3.2191088202426315,0.18359482658714038) node {};
\draw[color=black] (-0.0389164736016018,4.921990300092383) node {};
\draw [fill=uuuuuu] (0,0) circle (1pt);
\draw[color=uuuuuu] (0.07043111424851956,0.1744825275996303) node [below left=3pt]{$C$};
\draw[color=ffqqqq] (-0.27583624727686473,0.36584080633734206) node [above left=18pt]{$d_1$};
\draw[color=ffqqqq] (0.4531476717239443,-0.10799874101318223) node [below = 10.5pt]{$d_2$};
\draw[color=qqwuqq] (0.14332950614860046,0.22004402253718072) node [above left = 4pt]{$\theta_1$};
\draw[color=qqwuqq] (0.21622789804868137,-0.08066184405065198) node [below = 4pt]{$\theta_2$};
\draw [fill=uuuuuu] (-1.751712643960778,0) circle (1pt);
\draw[color=uuuuuu] (-1.6791302913534223,0.19270712557465047) node {};
\draw [fill=uuuuuu] (1.28,0) circle (1pt);
\draw[color=uuuuuu] (1.3552652714874454,0.19270712557465047) node {};
\draw[color=black] (-0.8407987845024917,-0.03510034911310157) node [above]{$x$};
\draw[color=black] (0.6718428474241871,-0.03510034911310157) node [above ] {$a-x$};
\draw [fill=uuuuuu] (-1.75,2.9166666666666665) circle (1pt);
\draw[color=uuuuuu] (-1.6791302913534223,3.0904182036028565) node {$A$};
\draw [fill=uuuuuu] (1.28,-1.28) circle (1pt);
\draw[color=uuuuuu] (1.3552652714874454,-1.1012393306517811) node [below=6pt]{$B$};
\draw[color=black] (-1.569782703503301,1.5686642726886728) node [left=0.5pt]{$y_1$};
\draw[color=black] (1.163906992749733,-0.527164494438646) node [right=0.5pt]{$y_2$};
\end{scriptsize}
\end{tikzpicture}
}
\newcommand{\primary}
{
\begin{tikzpicture}[line cap=round,line join=round,>=triangle 45,x=1cm,y=1cm, scale=0.8]

\clip(-6.208079428623365,-4.28103785801355) rectangle (7,3.030141435297047);
\draw [shift={(-2.23606797749979,2)},line width=0.5pt,color=qqwuqq,fill=qqwuqq,fill opacity=0.10000000149011612] (0,0) -- (138.18968510422138:0.28665495726070117) arc (138.18968510422138:180:0.28665495726070117) -- cycle;
\draw [shift={(-2.23606797749979,2)},line width=0.5pt,color=qqwuqq,fill=qqwuqq,fill opacity=0.10000000149011612] (0,0) -- (-41.810314895778596:0.28665495726070117) arc (-41.810314895778596:-11.8103148957786:0.28665495726070117) -- cycle;
\draw [shift={(2.850084796318772,0.9364916731037085)},line width=0.5pt,color=qqwuqq,fill=qqwuqq,fill opacity=0.10000000149011612] (0,0) -- (168.1896851042214:0.28665495726070117) arc (168.1896851042214:198.1896851042214:0.28665495726070117) -- cycle;
\draw [shift={(2.850084796318772,0.9364916731037085)},line width=0.5pt,color=qqwuqq,fill=qqwuqq,fill opacity=0.10000000149011612] (0,0) -- (-161.8103148957786:0.28665495726070117) arc (-161.8103148957786:-131.81031489577862:0.28665495726070117) -- cycle;
\draw [shift={(-0.6140168188189818,-2.9364916731037085)},line width=0.5pt,color=qqwuqq,fill=qqwuqq,fill opacity=0.10000000149011612] (0,0) -- (-143.6206297915572:0.28665495726070117) arc (-143.6206297915572:-101.81031489577859:0.28665495726070117) -- cycle;
\draw [shift={(6.086727470566947,2)},line width=0.5pt,color=qqwuqq,fill=qqwuqq,fill opacity=0.10000000149011612] (0,0) -- (-143.6206297915572:0.28665495726070117) arc (-143.6206297915572:0:0.28665495726070117) -- cycle;
\draw [shift={(-0.6140168188189818,-2.9364916731037085)},line width=0.5pt,color=qqwuqq,fill=qqwuqq,fill opacity=0.10000000149011612] (0,0) -- (48.18968510422141:0.28665495726070117) arc (48.18968510422141:78.18968510422141:0.28665495726070117) -- cycle;
\draw [line width=0.5pt,color=qqzzcc] (0,0) circle (3cm);
\draw [line width=0.5pt,color=ccqqqq,domain=-6.208079428623365:-2.23606797749979] plot(\x,{(-5.52786404500042-0*\x)/-2.76393202250021});
\draw [line width=0.5pt,color=ffqqqq,domain=-6.208079428623365:-0.6140168188189818] plot(\x,{(--5.256544921989533-1.55890472129316*\x)/-2.116041634551027});
\draw [line width=0.5pt,color=ffqqqq] (2.850084796318772,0.9364916731037085)-- (-0.6140168188189818,-2.9364916731037085);
\draw [line width=0.5pt,color=ffqqqq] (-2.23606797749979,2)-- (2.850084796318772,0.9364916731037085);
\draw [line width=0.5pt,dash pattern=on 1pt off 1pt,domain=-2.7300584533700087:8.430433722156438] plot(\x,{(-5.256544921989533--1.55890472129316*\x)/2.116041634551027});
\draw [line width=0.5pt,dash pattern=on 1pt off 1pt,domain=-5:8.430433722156438] plot(\x,{(--5.52786404500042-0*\x)/2.76393202250021});
\draw [line width=0.5pt,dash pattern=on 1pt off 1pt,domain=-6.208079428623365:0] plot(\x,{(-0--2*\x)/-2.23606797749979});
\draw [line width=0.5pt,dash pattern=on 1pt off 1pt,domain=-6.208079428623365:0] plot(\x,{(-0-2.9364916731037085*\x)/-0.6140168188189818});
\draw [rotate around={0:(0,0)},line width=0.5pt,color=zzffff,fill=zzffff,fill opacity=0.25] (0,0) ellipse (3cm and 3cm);
\draw [line width=0.5pt,dash pattern=on 1pt off 1pt] (0,0)-- (2.850084796318772,0.9364916731037085);
\draw [line width=0.5pt, draw=qqzzcc] (0,0) circle (3cm);
\begin{scriptsize}
\draw [fill=uuuuuu] (-2.23606797749979,2) circle (0.5pt);
\draw[color=uuuuuu] (-2.156689366005456,2.1850068384627845) node [anchor=south east]{$C$};
\draw [fill=uuuuuu] (0,0) circle (0.5pt);
\draw[color=uuuuuu] (0.07921930062801302,0.18797730287988548) node {$O$};
\draw[color=qqwuqq] (-2.166244531247479,2.156341342736714) node[ left=8pt] {$\alpha$};
\draw[color=qqwuqq] (-1.8986999044708248,2.0225690293483862) node [below right = 0.5pt]{$\beta$};
\draw [fill=uuuuuu] (2.850084796318772,0.9364916731037085) circle (0.5pt);
\draw[color=uuuuuu] (2.9266585427509777,1.1243834965981825) node [right] {$B$};
\draw[color=qqwuqq] (2.8693275512988374,1.0192766789359249) node [left = 6pt]{$\beta$};
\draw[color=qqwuqq] (2.8979930470249076,0.9428353569997373) node[below left= 6pt] {$\beta$};
\draw [fill=uuuuuu] (-0.6140168188189818,-2.9364916731037085) circle (0.5pt);
\draw[color=uuuuuu] (-0.541866440103506,-2.7454584264213104) node [anchor=east]{$A$};
\draw[color=qqwuqq] (-0.5132009443774359,-3.013003053197967) node [below left =6pt]{$\alpha$};
\draw [fill=uuuuuu] (6.086727470566947,2) circle (0.5pt);
\draw[color=uuuuuu] (6.165859559796901,2.1850068384627845) node {};
\draw[color=qqwuqq] (6.337852534153321,1.9079070464441048) node [below = 4pt]{$D(\alpha)$};
\draw[color=qqwuqq] (-0.37942863098910873,-2.7167929306952403) node [above= 4pt]{$\beta$};
\end{scriptsize}
\end{tikzpicture}
}

\newcommand{\primarycolor}
{
\begin{tikzpicture}[line cap=round,line join=round,>=triangle 45,x=1cm,y=1cm, scale=0.6]

\clip(-6.46924489473801,-6.191518174827493) rectangle (3.023218480966734,3.167438049025187);

\draw [rotate around={0:(0,0)},line width=0.25pt,color=ttffcc,fill=ttffcc,fill opacity=0.25] (0,0) ellipse (3cm and 3cm);

\draw [line width=0.5pt, draw=qqzzcc] (0,0) circle (3cm);

%red

\draw [line width=0.5pt, draw=myred] (-1.522457840938142,2.584980101)-- (2.7978098337965394,1.082709625851447);
\draw [line width=0.5pt,domain=-6.46924489473801:-1.522457840938142, draw=myred] plot(\x,{(-5.66929687567797-0*\x)/-2.193168478737922});
\draw [line width=0.5pt, draw=myred] (2.79780983379654,1.082709625851447)-- (0.6142485544560807,-2.9364432079215526);
\draw [line width=0.5pt,domain=-6.46924489473801:0.6142485544560807, draw=myred] plot(\x,{(--3.5519844101822113-0.9249782549023493*\x)/-1.0161333432758997});

% orange

\draw [line width=0.5pt, draw=myorange] (-1.5256042352006474,2.58312441)-- (2.7994604216279537,1.0784346747664604);
\draw [line width=0.5pt,domain=-6.46924489473801:-1.5256042352006474, draw=myorange] plot(\x,{(-5.657099504847531-0*\x)/-2.190022084475417});
\draw [line width=0.5pt, draw=myorange] (2.799460421627954,1.0784346747664604)-- (0.6016966061149287,-2.9390408629669946);
\draw [line width=0.5pt,domain=-6.46924489473801:0.6016966061149287, draw=myorange] plot(\x,{(--3.549434531558556-0.9218696303918201*\x)/-1.0189544287751555});

%yellow

\draw [line width=0.5pt, draw=myyellow] (-1.5290087088629778,2.581110685)-- (2.8012343878394748,1.0738183758837456);
\draw [line width=0.5pt,domain=-6.46924489473801:-1.5290087088629778, draw=myyellow] plot(\x,{(-5.6439020792788295-0*\x)/-2.1866176108130864});
\draw [line width=0.5pt, draw=myyellow] (2.8012343878394748,1.0738183758837456)-- (0.5881253151603842,-2.9417866363262135);
\draw [line width=0.5pt,domain=-6.46924489473801:0.5881253151603842, draw=myyellow] plot(\x,{(--3.5466675006620205-0.9185051879132415*\x)/-1.0219882402140028});

%green

\draw [line width=0.5pt, draw=mygreen] (-1.5334542960831044,2.578472013)-- (2.803532157795635,1.0678049635611138);
\draw [line width=0.5pt,domain=-6.46924489473801:-1.5334542960831044, draw=mygreen] plot(\x,{(-5.626669490386022-0*\x)/-2.18217202359296});
\draw [line width=0.5pt, draw=mygreen] (2.803532157795635,1.0678049635611138)-- (0.5704203050195396,-2.945270900209591);
\draw [line width=0.5pt,domain=-6.46924489473801:0.5704203050195396, draw=mygreen] plot(\x,{(--3.543041738976674-0.9141107372279382*\x)/-1.0259207101154784});

%blue

\draw [line width=0.5pt, draw=myblue] (-1.5389359535390779,2.575204095)-- (2.8063364904842576,1.0604128923096425);
\draw [line width=0.5pt,domain=-6.46924489473801:-1.5389359535390779, draw=myblue] plot(\x,{(-5.605421944423016-0*\x)/-2.1766903661369863});
\draw [line width=0.5pt, draw=myblue] (2.806336490484257,1.0604128923096425)-- (0.5486156646238993,-2.9494102550390098);
\draw [line width=0.5pt,domain=-6.46924489473801:0.5486156646238993, draw=myblue] plot(\x,{(--3.5385513393077326-0.9086906836594855*\x)/-1.0307244950954204});

%indigo

\draw [line width=0.5pt, draw=myindigo] (-1.5464879817929291,2.570675966)-- (2.810148103015898,1.05027026955739);
\draw [line width=0.5pt,domain=-6.46924489473801:-1.5464879817929291, draw=myindigo] plot(\x,{(-5.576151792125362-0*\x)/-2.169138337883135});
\draw [line width=0.5pt, draw=myindigo] (2.8101481030158975,1.05027026955739)-- (0.5186258999216984,-2.9548311586164124);
\draw [line width=0.5pt,domain=-6.46924489473801:0.5186258999216984, draw=myindigo] plot(\x,{(--3.532329301618131-0.9012216452518325*\x)/-1.0372614373863676});

%violet

\draw [line width=0.5pt, draw=myviolet] (-1.5566119463587236,2.564558295)-- (2.8151645805619148,1.0367489495290845);
\draw [line width=0.5pt,domain=-6.46924489473801:-1.5566119463587236, draw=myviolet] plot(\x,{(-5.536918220115211-0*\x)/-2.1590143733173406});
\draw [line width=0.5pt, draw=myviolet] (2.8151645805619148,1.0367489495290845)-- (0.4785192094328503,-2.9615906817458355);
\draw [line width=0.5pt,domain=-6.46924489473801:0.4785192094328503, draw=myviolet] plot(\x,{(--3.5239230968623514-0.8912072824037787*\x)/-1.045878254458771});

\end{tikzpicture}
}

\newcommand{\primarycolorpers}
{
\begin{tikzpicture}[line cap=round,line join=round,>=triangle 45,x=1cm,y=1cm,scale=1.2]
\clip(-1.6481524516351462,-0.183021382422293) rectangle (5.512524834546647,3.8);
\draw [line width=2pt,domain=-1.6481524516351462:5.512524834546647] plot(\x,{(-0-0*\x)/1});
\draw[line width=0.75pt,draw=myviolet, smooth,samples=100,domain=0:5.512524834546647] plot(\x,{0.6145562845*(\x)});
\draw[line width=0.75pt,draw=myindigo, smooth,samples=100,domain=0:5.512524834546647] plot(\x,{0.620945365*(\x)});
\draw[line width=0.75pt,draw=myblue, smooth,samples=100,domain=0:5.512524834546647] plot(\x,{0.6257073969*(\x)});
\draw[line width=0.75pt,draw=mygreen, smooth,samples=100,domain=0:5.512524834546647] plot(\x,{0.6291615751*(\x)});
\draw[line width=0.75pt,draw=myyellow, smooth,samples=100,domain=0:5.512524834546647] plot(\x,{0.6319613238*(\x)});
\draw[line width=0.75pt,draw=myorange, smooth,samples=100,domain=0:5.512524834546647] plot(\x,{0.6341043966*(\x)});
\draw[line width=0.75pt,draw=myred, smooth,samples=100,domain=0:5.512524834546647] plot(\x,{0.636084205*(\x)});
\begin{scriptsize}
\draw [fill=uuuuuu] (0,0) circle (2.5pt);
\draw[color=uuuuuu] (-0.07955947629757938,0.12208791120282789) node [above left]{observer};
\end{scriptsize}
\end{tikzpicture}
}

\newcommand{\secondary}
{
\begin{tikzpicture}[line cap=round,line join=round,>=triangle 45,x=1cm,y=1cm,scale=0.75]
\clip(-8,-5.5) rectangle (3.430433722156438,3.330141435297047);
\draw [shift={(-1.0770329614269012,-2.8)},line width=0.5pt,color=qqwuqq,fill=qqwuqq,fill opacity=0.10000000149011612] (0,0) -- (180:0.5576249556023645) arc (180:248.96053021868272:0.5576249556023645) -- cycle;
\draw [shift={(-1.0770329614269012,-2.8)},line width=0.5pt,color=qqwuqq,fill=qqwuqq,fill opacity=0.10000000149011612] (0,0) -- (24.51948074501533:0.5576249556023645) arc (24.51948074501533:68.96053021868276:0.5576249556023645) -- cycle;
\draw [shift={(2.820479778289686,-1.022200479484804)},line width=0.5pt,color=qqwuqq,fill=qqwuqq,fill opacity=0.10000000149011612] (0,0) -- (160.0784312713479:0.5576249556023645) arc (160.0784312713479:204.51948074501533:0.5576249556023645) -- cycle;
\draw [shift={(2.820479778289686,-1.022200479484804)},line width=0.5pt,color=qqwuqq,fill=qqwuqq,fill opacity=0.10000000149011612] (0,0) -- (115.63738179768048:0.5576249556023645) arc (115.63738179768048:160.0784312713479:0.5576249556023645) -- cycle;
\draw [shift={(0.9669788773499723,2.839885886925562)},line width=0.5pt,color=qqwuqq,fill=qqwuqq,fill opacity=0.10000000149011612] (0,0) -- (-108.80366767598697:0.5576249556023645) arc (-108.80366767598697:-64.36261820231954:0.5576249556023645) -- cycle;
\draw [shift={(0.9669788773499723,2.839885886925562)},line width=0.5pt,color=qqwuqq,fill=qqwuqq,fill opacity=0.10000000149011612] (0,0) -- (-153.2447171496544:0.5576249556023645) arc (-153.2447171496544:-108.80366767598697:0.5576249556023645) -- cycle;
\draw [shift={(-2.858210938600653,0.9113891761830262)},line width=0.5pt,color=qqwuqq,fill=qqwuqq,fill opacity=0.10000000149011612] (0,0) -- (162.31423337667817:0.5576249556023645) arc (162.31423337667817:231.27476359536095:0.5576249556023645) -- cycle;
\draw [shift={(-5.8342792390694145,-2.8)},line width=0.5pt,color=qqwuqq,fill=qqwuqq,fill opacity=0.10000000149011612] (0,0) -- (0:0.5576249556023645) arc (0:231.27476359536095:0.5576249556023645) -- cycle;
\draw [line width=0.5pt,color=qqzzff] (0,0) circle (3cm);
\draw [line width=0.5pt,dash pattern=on 1pt off 1pt,domain=-12.729178337151982:0] plot(\x,{(-0-2.8*\x)/-1.0770329614269012});
\draw [line width=0.5pt,dash pattern=on 1pt off 1pt] (0,0)-- (2.820479778289686,-1.022200479484804);
\draw [line width=0.5pt,dash pattern=on 1pt off 1pt,domain=-12.729178337151982:0] plot(\x,{(-0--0.9113891761830262*\x)/-2.858210938600653});
\draw [line width=0.5pt,dash pattern=on 1pt off 1pt] (0,0)-- (0.9669788773499723,2.839885886925562);
\draw [line width=0.5pt,color=ffqqqq] (-1.0770329614269012,-2.8)-- (2.820479778289686,-1.022200479484804);
\draw [line width=0.5pt,color=ffqqqq] (2.820479778289686,-1.022200479484804)-- (0.9669788773499723,2.839885886925562);
\draw [line width=0.5pt,color=ffqqqq] (0.9669788773499723,2.839885886925562)-- (-2.858210938600653,0.9113891761830262);
\draw [line width=0.5pt,color=ffqqqq,domain=-12.729178337151982:-1.0770329614269012] plot(\x,{(--8.17056619074427-0*\x)/-2.9180593538372395});
\draw [line width=0.5pt,color=ffqqqq,domain=-12.729178337151982:-2.858210938600653] plot(\x,{(-5.52727962259094-1.5400480331769655*\x)/-1.2349252301939693});
\draw [rotate around={0:(0,0)},line width=0.5pt,color=ttffcc,fill=ttffcc,fill opacity=0.25] (0,0) ellipse (3cm and 3cm);
\begin{scriptsize}
\draw [fill=black] (0,0) circle (2.5pt);
\draw[color=black] (0.337833122463426,0.2911964584027634) node {$O$};
\draw [fill=uuuuuu] (-1.0770329614269012,-2.8) circle (0.5pt);
\draw[color=uuuuuu] (-0.9632917739420913,-2.9616157826110254) node {$A$};
\draw[color=qqwuqq] (-1.3722167413838253,-3.147490767811813) node [below left]{$\alpha$};
\draw[color=qqwuqq] (-0.21979183313893852,-2.088003352167322) node [left]{$\beta$};
\draw [fill=uuuuuu] (2.820479778289686,-1.022200479484804) circle (0.5pt);
\draw[color=uuuuuu] (3.2560703901158004,-0.9355784439224368) node {$B$};
\draw[color=qqwuqq] (2.3266954641118596,-0.8612284498421217) node [below left]{$\beta$};
\draw[color=qqwuqq] (2.5311579478327264,-0.28501599571967917) node [below left]{$\beta$};
\draw [fill=uuuuuu] (0.9669788773499723,2.839885886925562) circle (0.5pt);
\draw[color=uuuuuu] (1.1185080603067363,3.209433726055134) node {$C$};
\draw[color=qqwuqq] (1.5460205262685491,2.149946310410643) node [left = 10pt]{$\beta$};
\draw[color=qqwuqq] (0.7653455884252388,2.2800588000511945) node [ left = 4pt]{$\beta$};
\draw [fill=uuuuuu] (-2.858210938600653,0.9113891761830262) circle (0.5pt);
\draw[color=uuuuuu] (-3.0450916081909187,1.4622088651677274) node {$D$};
\draw[color=qqwuqq] (-3.435429077112574,0.867408912525206) node [left = 2pt]{$\alpha$};
\draw [fill=uuuuuu] (-5.8342792390694145,-2.8) circle (0.5pt);
\draw[color=uuuuuu] (-5.53581640988148,-2.998790779651183) node {$E$};
\draw[color=qqwuqq] (-5.888978881762978,-1.9021283669665343) node []{$D(\alpha)$};
\end{scriptsize}
\end{tikzpicture}
}

\newcommand{\secondarycolor}
{
\begin{tikzpicture}[line cap=round,line join=round,>=triangle 45,x=1cm,y=1cm,scale=1.2]
\clip(-1.6481524516351462,-0.183021382422293) rectangle (5.512524834546647,6.1);
\draw [line width=2pt,domain=-1.6481524516351462:5.512524834546647] plot(\x,{(-0-0*\x)/1});
\draw[line width=0.75pt,draw=myviolet, smooth,samples=100,domain=0:5.512524834546647] plot(\x,{0.6145562845*(\x)});
\draw[line width=0.75pt,draw=myindigo, smooth,samples=100,domain=0:5.512524834546647] plot(\x,{0.620945365*(\x)});
\draw[line width=0.75pt,draw=myblue, smooth,samples=100,domain=0:5.512524834546647] plot(\x,{0.6257073969*(\x)});
\draw[line width=0.75pt,draw=mygreen, smooth,samples=100,domain=0:5.512524834546647] plot(\x,{0.6291615751*(\x)});
\draw[line width=0.75pt,draw=myyellow, smooth,samples=100,domain=0:5.512524834546647] plot(\x,{0.6319613238*(\x)});
\draw[line width=0.75pt,draw=myorange, smooth,samples=100,domain=0:5.512524834546647] plot(\x,{0.6341043966*(\x)});
\draw[line width=0.75pt,draw=myred, smooth,samples=100,domain=0:5.512524834546647] plot(\x,{0.636084205*(\x)});
\draw[line width=0.75pt,draw=myviolet,smooth,samples=100,domain=0:7] plot(\x,{0.7544984823*(\x)});
\draw[line width=0.75pt,draw=myindigo,smooth,samples=100,domain=0:7] plot(\x,{0.7451804338*(\x)});
\draw[line width=0.75pt,draw=myblue,smooth,samples=100,domain=0:7] plot(\x,{0.7380829482*(\x)});
\draw[line width=0.75pt,draw=mygreen,smooth,samples=100,domain=0:7] plot(\x,{0.7328511044*(\x)});
\draw[line width=0.75pt,draw=myyellow,smooth,samples=100,domain=0:7] plot(\x,{0.7285578515*(\x)});
\draw[line width=0.75pt,draw=myorange,smooth,samples=100,domain=0:7] plot(\x,{0.7252392408*(\x)});
\draw[line width=0.75pt,draw=myred,smooth,samples=100,domain=0:7] plot(\x,{0.7221482199*(\x)});
\begin{scriptsize}
\draw [fill=uuuuuu] (0,0) circle (2.5pt);
\draw[color=uuuuuu] (-0.07955947629757938,0.12208791120282789) node [above left]{observer};
\end{scriptsize}
\end{tikzpicture}
}

\title{The Calculus of Rainbows}
\author{
Matt Alejo
\and
Joshua de Luna
\and
Althea Loyola
}
\date{7 June 2021}


% Document proper
\begin{document}

\definecolor{zzffff}{rgb}{0.6,1,1}
\definecolor{ffqqqq}{rgb}{1,0,0}
\definecolor{qqwuqq}{rgb}{0,0.39215686274509803,0}
\definecolor{ccqqqq}{rgb}{0.8,0,0}
\definecolor{uuuuuu}{rgb}{0.26666666666666666,0.26666666666666666,0.26666666666666666}
\definecolor{qqzzcc}{rgb}{0,0.6,1}
\definecolor{ttffcc}{rgb}{0.2,1,0.8}
\definecolor{xdxdff}{rgb}{0.49019607843137253,0.49019607843137253,1}
\definecolor{qqzzff}{rgb}{0,0.6,1}
\definecolor{myred}{HTML}{B53737}
\definecolor{myorange}{HTML}{FF6600}
\definecolor{myyellow}{HTML}{FFD400}
\definecolor{mygreen}{HTML}{006400}
\definecolor{myblue}{HTML}{1C4966}
\definecolor{myindigo}{HTML}{4B0082}
\definecolor{myviolet}{HTML}{3c1361}


\setstretch{1.05}

\maketitle

\tableofcontents

\section{Introduction}

A rainbow is a meteorological phenomenon that usually occurs after rain, characterized by a circular spectrum of light appearing in the sky. Because of its beauty and the difficulty of explaining the phenomenon, rainbows have a place in legends in different cultures. In Greco-Roman mythology, the rainbow was considered to be a path made by Iris, a messenger, between Earth and Heaven. In the Bible's Genesis flood narrative God put the rainbow in the sky after flooding the world to wash away sin as the sign of his promise that he would never again destroy the earth with a flood.\\

With scientific breakthroughs and discoveries throughout the history of humanity, the legends that attempt to explain the existence of rainbows are relegated to creative literature, while rainbows are explained to be optical illusions due to the reflection and refraction of light. This paper goes through the scientific explanation of rainbows and uses Calculus and Physics to derive the underlying mathematics from the phenomenon of rainbows.

\section{Scientific concepts}

\subsection{Fermat's principle of optics} \label{fermat}

Fermat's principle states that the path taken by a ray between two given points
is the path that can be traversed in the least time. In an ideal scenario -- in a vacuum -- the path of least time is also the path of least distance, the line, as descibed by the equation

\begin{equation} \label{distance}
d = vt
\end{equation}

where $d$ is the distance, $v$ is the velocity, and $t$ is the time.\\

However, when there are interferences to the path of the light, the path may take a different path, not a straight line as one may presume, but the path of least time as suggested by Fermat.

\subsection{Snell's law} \label{snells}

The first case of Fermat's principle of optics is when light passes through two different objects \cite{hilburnnd}. The phase velocity of light is computed as

\begin{equation} \label{phase}
v = \dfrac{c}{n}
\end{equation}

where $v$ is the phase velocity of light, $c$ is the velocity of light in a vacuum, and $n$ is the refractive index of the medium the light is passing through.\\

Now, we consider a light ray travelling from point $A$ to point $B$ with different refractive indices, $n_1$ and $n_2$, respectively.\\

\begin{figure}[H]
\centering
\snells
\caption{Refraction of light passing through different mediums.}
\label{fig:snell}
\end{figure}

Figure \ref{fig:snell} visualizes how light behaves when there is a change in the medium it travels to The calculation can be visualized using the figure below. Using the Pythagorean Theorem,\\

\begin{align}
d_1 &= \sqrt{x^2 + y_1^2} \nonumber\\
d_2 &= \sqrt{(a-x)^2 +  y_2^2} \nonumber
\end{align}

From Equation \eqref{distance},

\begin{align}
d_1 &= v_1 t_1 & d_2 &= v_2 t_2 \nonumber\\
t_1 &= \dfrac{d_1}{v_1} & t_2 &= \dfrac{d_2}{v_2}\nonumber\\
t_1 &= \dfrac{\sqrt{x^2 + y_1^2}}{v_1} & t_2 &= \dfrac{\sqrt{(a-x)^2 + y_2^2}}{v_2}
\end{align}

We can then create a function $T(x)$ that measures the total time depending on the value of $x$.

\begin{equation}
T(x) = t_1 + t_2 =\dfrac{\sqrt{(a-x)^2 + y_2^2}}{v_1} + \dfrac{\sqrt{(a-x)^2 + y_2^2}}{v_2} 
\end{equation}

Getting the derivative of $T$,

\begin{align}
T'(x) &= (2x) \left( \dfrac{1}{2} \right) \left(\dfrac{(x^2 + y_1^2)^{-1/2}}{v_1} \right) - 2(a-x)\left( \dfrac{1}{2} \right) \left( \dfrac{[(a-x)^2+y_2^2]^{-1/2}}{v_2} \right) \nonumber\\
T'(x) &= (x) \dfrac{(x^2 + y_1^2)^{-1/2}}{v_1} - (a-x)\dfrac{[(a-x)^2+y_2^2]^{-1/2}}{v_2}
\end{align}

As mentioned in Section \ref{fermat}, light rays take the path of least time. Thus we can get the minimum time by equating $T'(x) = 0$.

\begin{align}
(x) \dfrac{(x^2 + y_1^2)^{-1/2}}{v_1} - (a-x)\dfrac{[(a-x)^2+y_2^2]^{-1/2}}{v_2} &= 0  \nonumber \\
\dfrac{1}{v_1} \dfrac{x}{(x^2 + y_1^2)^{1/2}}  -  \dfrac{1}{v_2} \dfrac{a-x}{[(a-x)^2 + y_2^2]^{1/2}}  &= 0 \label{badsnell}
\end{align}

Observe from the figure that

\begin{align}
\sin \theta_1 &= \dfrac{x}{d_1} = \dfrac{x}{\sqrt{x^2 + y_1^2}} =\dfrac{x}{(x^2 +y_1^2)^{1/2}} \\
\sin \theta_2 &= \dfrac{a-x}{d_2} = \dfrac{a-x}{\sqrt{(a-x)^2 + y_2^2}} = \dfrac{a-x}{[(a-x)^2 + y_2^2]^{1/2}} 
\end{align}

thus, when substituted to Equation \eqref{badsnell}, gives us

\begin{align} 
\dfrac{1}{v_1}\dfrac{x}{(x^2 + y_1^2)^{1/2}} - \dfrac{1}{v_2} \dfrac{a-x}{[(a-x)^2 + y_2^2]^{1/2}} &= 0 \nonumber\\
\sin(\theta_1)\dfrac{1}{v_1} - \sin(\theta_2)\dfrac{1}{v_2} &= 0 \label{protosnell}
\end{align}

Using Equation \eqref{phase},

\begin{align}
v_1 &= \dfrac{c}{n_1} & v_2 &= \dfrac{c}{n_1} \nonumber\\
\dfrac{1}{v_1} &= \dfrac{n_1}{c} & \dfrac{1}{v_2} &= \dfrac{n_2}{c}
\end{align}

Substituting these to Equation \eqref{protosnell},

\begin{align}
\sin(\theta_1)\dfrac{1}{v_1} - \sin(\theta_2)\dfrac{1}{v_2} &= 0 \nonumber\\
\sin(\theta_1)\dfrac{n_1}{c} - \sin(\theta_2)\dfrac{n_2}{c} &= 0 \nonumber\\
\sin(\theta_1)\dfrac{n_1}{c} &= \sin(\theta_2)\dfrac{n_2}{c} \nonumber\\
n_1 \sin(\theta_1) &= n_2 \sin(\theta_2) \label{refract}
\end{align}

Equation \eqref{refract} is what we call the law of refraction or {\itshape Snell's law}. Snell's Law shows the relationship between the incident angle and the refracted angle in two different mediums \cite{flens20}. 

\subsection{Law of reflection} \label{reflection}

Another case of Fermat's principle of optics is if rather than passing through, the light ray bounces back.\\


\begin{figure}[H]
\centering
\reflection
\caption{A light ray getting reflected at a flat surface}
\label{fig:reflect}
\end{figure}

Figure \ref{fig:reflect} shows a ray of light from point $A$ reflects off the surface at point $P$ before arriving at point $B$, a horizontal distance $a$ from point $A$.\\

Using the Pythagorean theorem,

\begin{align}
d_1 &= \sqrt{x^2 + y_1^2} \nonumber\\
d_2 &= \sqrt{(a-x)^2 +y_2^2} \nonumber
\end{align}

Using Equation \eqref{distance},

\begin{align}
d_1 &= vt_1 & d_2 &= vt_2 & \text{ note that the velocity of light at } d_1 \text{ and } d_2 \text{ are equal} \nonumber\\
t_1 &= \dfrac{d_1}{v} & t_2 &= \dfrac{d_2}{v}
\end{align}

We can then create a function $T(x)$ that measures the total time depending on the value of $x$

\begin{equation} 
T(x) = t_1 + t_2 = \dfrac{\sqrt{x^2 +y_1^2}}{v} + \dfrac{\sqrt{(a-x)^2 + y_2^2}}{v}
\end{equation}

Getting the derivative of $T$,

\begin{equation}
T'(x) = \dfrac{x}{v\sqrt{x^2 + y_1^2}} + \dfrac{-(a-x)}{v\sqrt{(a-x)^2 + y_2^2}}
\end{equation}

As mentioned Section \ref{fermat}, light rays take the path of least time, thus equating $T'(x) = 0$ gets the least time.

\begin{align}
\dfrac{x}{v\sqrt{x^2 + y_1^2}} + \dfrac{-(a-x)}{v\sqrt{(a-x)^2 + y_2^2}} &= 0 \nonumber\\
\dfrac{x}{v\sqrt{x^2 + y_1^2}} &= \dfrac{a-x}{v\sqrt{(a-x)^2 + y_2^2}}\nonumber\\
\dfrac{x}{\sqrt{x^2 + y_1^2}} &= \dfrac{a-x}{\sqrt{(a-x)^2 + y_2^2}} \label{protoreflect}
\end{align}

Observe from Figure \ref{fig:reflect} that

\begin{align}
\sin \theta_1 &= \dfrac{x}{\sqrt{x^2 + y_1^2}} \\
\sin \theta_2 &= \dfrac{a-x}{\sqrt{(a-x)^2 + y_2^2}} 
\end{align}

Substituting these to Equation \eqref{protoreflect},

\begin{align}
\sin \theta_1 &= \sin \theta_2 \nonumber\\
\theta_1 &= \theta_2 \label{reflect}
\end{align}

Equation \eqref{reflect} is what we call the law of reflection, showing that the angle of incidence $\theta_1$ is equal to the angle of reflection $\theta_2$.\\

Rough surfaces contain varying planes that diffuses light at different angles at the point where the light ray bounces. This is what we call {\itshape diffused reflection} \cite{refractionnd}.  On the other hand, when light is directed in a smooth plane like polished metal or a glass mirror, the light ray is reflected at an angle equal to the incident angle with respect to the line perpendicular to the plane. This is what we call {\itshape specular reflection} \cite{reflectionnd}.

\subsection{Light wavelengths and their refractive interactions} \label{indices}

\definecolor{myred}{HTML}{B53737}
\definecolor{myorange}{HTML}{FF6600}
\definecolor{myyellow}{HTML}{FFD400}
\definecolor{mygreen}{HTML}{006400}
\definecolor{myblue}{HTML}{1C4966}
\definecolor{myindigo}{HTML}{4B0082}
\definecolor{myviolet}{HTML}{3c1361}

Using mathematical tools, Descartes had an objective of improving the explanation of rainbows. By studying the passage of a light ray through a glass sphere with water inside. “He concluded that the primary arc was created by a single reflection inside the drop, while the secondary one may be caused by two inner reflections.” \cite[9]{corradi16} Descartes understood roughly why the rainbow is located where it is, but he was fairly straightforward in declaring that he didn't understand why the rainbow showed different colors. \\

Isaac Newton discovered that different wavelengths have different refractive index, an important piece of information Descartes lacked resulting on his inconclusive explanation of the different colors of the rainbow \cite{casselman09}. According to Newton, the decomposition of light causes the production of “iridescent figures” by the prisms. Newton was also able to illustrate that blue light is refracted more than red, explaining the main components of the rainbow \cite[9]{corradi16}.


\begin{table}[H]
\centering
\def\arraystretch{1.25}
\caption{Visible light wavelengths and their corresponding refractive indices on water \cite{lavennd}}
\begin{tabular}{|c|c|c|}
\hline
{\bfseries Wavelength (nm)} & {\bfseries Refractive Index} & {\bfseries Color}  \\ \hline
400             & 1.3445           & \textcolor{myviolet}{Violet} \\ \hline
450             & 1.3406           & \textcolor{myindigo}{Indigo} \\ \hline
500             & 1.3377           & \textcolor{myblue}{Blue}   \\ \hline
550             & 1.3356           & \textcolor{mygreen}{Green}  \\ \hline
600             & 1.3339           & \textcolor{myyellow}{Yellow} \\ \hline
650             & 1.3326           & \textcolor{myorange}{Orange} \\ \hline
700             & 1.3314           & \textcolor{myred}{Red}    \\ \hline
\end{tabular}
\end{table}


\section{Mathematical derivation}

\subsection{The angle of a primary rainbow} \label{sec:primary}

\begin{figure}[H]
\centering
\primary
\caption{Formation of a primary rainbow, showing Snell’s Law and the Law of Reflection at work.}
\label{primary}
\end{figure}


Figure \ref{primary} shows a ray of sunlight entering a spherical raindrop at $A$. The figure assumes ideal conditions such as the raindrops being perfectly spherical and that the light rays meet no interference. As the light ray enters the raindrop, some of the light is reflected, but the line $AB$ shows the path of the part that enters the drop. \\

Notice that the light is refracted toward the normal line $AO$. Since there is a change in the medium of the passing light, Snell’s law applies. 

\begin{equation} \label{eq:snell}
\sin \alpha = k \sin \beta
\end{equation}

where $\alpha$ is the angle of incidence, $\beta$ is the angle of refraction, and $k \approx 4/3$ is the approximate ratio between the index of refraction of water and the index of refraction of air. Note that the angle of refraction of air is very close to 1, so it can be assumed that $k$ is the index of refraction of water.\\

At $B$ some of the light passes through the drop and is refracted into the air, but some of the light is also internally reflected, as represented by $BC$. Note that the angle of incidence equals the angle of reflection. When the ray reaches $C$, part of it is reflected, but for the time being, we are more interested in the part that leaves the raindrop at $C$. (Notice that it is refracted away from the normal line.) \\

The angle of deviation $D(\alpha)$ is the amount of clockwise rotation that the ray has undergone during the stage of refraction, internal reflection, and another refraction. Using Geometry, 

\begin{align}
D(\alpha) &= (\alpha - \beta) + (\pi - 2\beta) +(\alpha - \beta) \nonumber\\
D(\alpha) &= \pi + 2\alpha - 4\beta
\end{align}

Using Equation \eqref{eq:snell} to isolate $\beta$ in terms of $\alpha$,

\begin{align}
\sin \alpha &= k \sin \beta \nonumber\\
\dfrac{\sin \alpha}{k} &= \sin \beta \nonumber\\
\beta &= \sin^{-1} \left(\dfrac{\sin \alpha}{k} \right) \label{beta}
\end{align}

Substituting $\beta$ on Equation \eqref{beta},

\begin{equation}
D(\alpha) = \pi + 2\alpha - 4\sin^{-1} \left( \dfrac{\sin \alpha}{k} \right)
\end{equation}

Getting the derivative of $D$,

\begin{align}
D’(\alpha) &= 0 + 2 - 4 \left[ \frac{1}{ \sqrt{1 -  \left( \dfrac{\sin \alpha}{k} \right) ^2 } } \right] \left(  \dfrac{\cos \alpha}{k} \right) \nonumber \\
D(\alpha) &= 2 - \dfrac{4 \cos \alpha}{k \sqrt{1 -  \dfrac{\sin^2 \alpha}{k^2}}}
\end{align}

To get the minimum of $D(\alpha)$, we shall equate $D(\alpha)$ to $0$.

\begin{align}
2 - \dfrac{4 \cos \alpha}{k \sqrt{1 -  \dfrac{\sin^2 \alpha}{k^2}}} &= 0 \nonumber\\
2 &=  \dfrac{4 \cos \alpha}{k \sqrt{1 -  \dfrac{\sin^2 \alpha}{k^2}}}
\end{align}

It is good to note that since $k>1$, then $1-\dfrac{\sin^2 \alpha}{k} >0$ for all $\alpha$.

\begin{align}
2 k \sqrt{1 -  \dfrac{\sin^2 \alpha}{k^2}} &= 4 \cos \alpha \nonumber\\
k \sqrt{1 -  \dfrac{\sin^2 \alpha}{k^2}} &= 2\cos \alpha
\end{align}

Squaring both sides,

\begin{align}
k^2 \left(1 -  \dfrac{\sin^2 \alpha}{k^2} \right) &= 4 \cos^2 \alpha \nonumber\\
k^2 - \sin^2 \alpha &= 4 \cos^2 \alpha \nonumber\\
k^2 &= \sin^2 \alpha + 4 \cos^2 \alpha \nonumber\\
k^2 &= \sin^2 \alpha + \cos^2 \alpha + 3\cos^2 \alpha 
\end{align}

Note that $\sin^2 \alpha + \cos^2 \alpha = 1$.

\begin{align}
k^2 &= 1 + 3\cos^2 \alpha \nonumber\\
\sqrt{\dfrac{k^2 -1}{3}} &= \cos \alpha \nonumber\\
\alpha &=  \cos^{-1} \sqrt{\dfrac{k^2 -1}{3}}
\end{align}

$D(\alpha)$ is minimum when 

\begin{equation}\label{min}
\alpha =  \cos^{-1} \sqrt{\dfrac{k^2 -1}{3}}
\end{equation}

Substituting $k \approx \dfrac{4}{3}$,

\begin{align*}
\alpha &\approx  \cos^{-1} \sqrt{\dfrac{(4/3 )^2 -1}{3}} \\
&\approx \cos^{-1} \sqrt{\dfrac{16/9 -1}{3}}\\
&\approx \cos^{-1} \sqrt{\dfrac{7}{27}}\\
\alpha &\approx 1.0366\\
\alpha &\approx 59.4 \degree
\end{align*}

$D(\alpha)$ is minimum at $\alpha \approx 59.4 \degree$.\\

Substituting $\alpha \approx  59.4 \degree$,

\begin{align*}
\min(D(\alpha)) \approx D(1.0366) &\approx \pi + 2\alpha - 4\sin^{-1} \left( \dfrac{\sin \alpha}{4/3} \right)\\
&\approx \pi + 2(1.0366) -  4\sin^{-1} \left( \dfrac{\sin 1.0366}{4/3} \right)\\
&\approx 138 \degree
\end{align*}

The minimum value of the deviation is $D(\alpha) \approx 138 \degree$ which occurs when $\alpha \approx 59.4 \degree$.\\

The minimum deviation angle has huge significance since $D’(\alpha) \approx 0$ implies that $\Delta D/\Delta \alpha \approx 0$, which means that light rays deviate by approximately the same amount at $\alpha \approx 59.4 \degree$. This implies that there is a concentration of rays coming from near the direction of the minimum deviation, causing the formation of the rainbow from these concentrated light rays \cite[219]{stewart21}.\\

The figure shows that the angle of elevation from the observer up to the highest point on the rainbow is $180 \degree - 138 \degree = 42 \degree$. This angle is called the {\itshape rainbow angle}.

\subsection{Colors in a primary rainbow} \label{sec:primarycolor}

Sunlight comprises a range of wavelengths, from the reds with the longest wavelengths and the violets with the shortest wavelengths. As Newton discovered in his prism experiments of
1666, the index of refraction is different for each color in a phenomenon called dispersion \cite[219]{stewart21}.
For red light, the refractive index is $k \approx 1.3318$, whereas for violet light it is $k \approx 1.3435$.\\

As previously mentioned in Equation \eqref{min}, the minimum of $D(\alpha)$ is when $\alpha = \cos^{-1} \sqrt{\dfrac{k^2 -1}{3}}$\\

Substituting $k \approx 1.3318$ for red light,

\begin{align*}
\alpha &= \cos^{-1} \sqrt{\dfrac{1.3318^2 -1}{3}}\\
&\approx 1.0381
\end{align*}

Substituting $\alpha \approx 1.0381$ and $k \approx 1.3318$ on $D(\alpha)$,

\begin{align*}
D(1.0381) &= \pi + 2(1.0381) - 4\sin^{-1} \left( \dfrac{\sin 1.0381}{1.3318} \right)\\
&\approx 2.4041\\
&\approx 137.7 \degree
\end{align*}

The rainbow angle of the red light is $\approx 180 \degree - 137.7 \degree \approx 42.3 \degree$.\\

Substituting $k \approx 1.3435$ for violet light,

\begin{align*}
\alpha &= \cos^{-1} \sqrt{\dfrac{1.3435^2 -1}{3}}\\
&\approx 1.0263
\end{align*}

Substituting $\alpha \approx 1.0263$ and $k \approx 1.3435$ on $D(\alpha)$,

\begin{align*}
D(1.0263) &= \pi + 2(1.0263) - 4\sin^{-1} \left( \dfrac{\sin 1.0263}{1.3435} \right)\\
&\approx 2.4334\\
&\approx 139.4 \degree
\end{align*}

The rainbow angle of the violet light is $\approx 180 \degree - 139.4 \degree = 40.6 \degree$.\\


\begin{figure}[H]
\centering
\primarycolor
\caption{Colors of the rainbow produced in a raindrop based on their respective minimum $D(\alpha)$ value and refractive indices. Note that color produced is violet (on top) and red (on the bottom).}
\label{fig:primarycolor}
\end{figure}

\begin{figure}[H]
\centering
\primarycolorpers
\caption{However, from the perspective of a person, each wavelength of light of the rainbow comes from different droplets of light, thus the rainbow seen by an observer in the ground is red (on top) and violet (on the bottom)}
\label{fig:primarycolor2}
\end{figure}

As observed in the calculations and as visualized in Figure \ref{fig:primarycolor2}, the rainbow angle of the red light ($42.3 \degree$) is larger than the rainbow angle of the violet light ($40.6 \degree$). This implies that the rainbow’s color appears red on top, violet on the bottom, and all other visible light wavelengths in between.

\subsection{Double rainbows} \label{sec:secondary}

There are instances where another fainter rainbow appears in addition of the stronger rainbow. This phenomenon is called {\itshape double rainbow}, and the fainter rainbow is colloquially known as the {\itshape secondary rainbow}. 

\begin{figure}[H]
\centering
\secondary
\caption{Formation of a secondary rainbow. Compared to the primary rainbow, the light ray is internally reflected twice.}
\label{fig:secondary}
\end{figure}

As visulized in Figure \ref{fig:secondary}, this phenomenon results from the part of a ray that enters a raindrop and is refracted at $A$, reflected twice (at $B$ and $C$), and refracted as it leaves the drop at $D$. Whereas $D(\alpha)$ is the amount of {\itshape clockwise} deviation, in secondary rainbows, the deviation angle $D(\alpha)$ refers to the total amount of counterclockwise rotation that the ray undergoes in this four-stage process.\\

As seen in Figure \ref{fig:secondary}, the light ray has undergone two refractions and two internal reflections. The refractions deviate the ray by $\alpha - \beta$ while the internal reflections deviate the ray by $\pi - 2\beta$. Therefore, the angle of deviation $D(\alpha)$ is

\begin{align}
D(\alpha) &= (\alpha - \beta) + (\pi - 2\beta) + (\pi - 2\beta) + (\alpha - \beta) \nonumber \\
D(\alpha) &= 2\pi + 2\alpha - 6\beta \label{eq:secondary}
\end{align}

Using Equation \eqref{eq:snell} to get $\beta$ in terms of $\alpha$,

\begin{align*}
\sin \alpha &= k \sin \beta\\
\beta &= \sin^{-1} \left( \frac{\sin \alpha}{k} \right)
\end{align*}

Substuting $\beta$ on Equation \eqref{eq:secondary},

\begin{align}
D(\alpha) &= 2\pi + 2\alpha - 6\beta \nonumber\\
D(\alpha) &= 2\pi + 2\alpha - 6 \sin^{-1} \left( \frac{\sin \alpha}{k} \right)
\end{align}

In order to get the minimum value of $D$, we can equate its derivative, $D'(\alpha) = 0$.

\begin{align}
D'(\alpha) &= 0 + 2 - 6 \left[ \frac{1}{\sqrt{1- \left( \dfrac{\sin \alpha}{k}\right)^2}}\right]\left(\dfrac{\cos \alpha}{k}\right) \nonumber\\
D'(\alpha) &= 2 - \dfrac{6\cos \alpha}{k\sqrt{1-\dfrac{\sin^2 \alpha}{k^2}}} \label{eq:derivativesec}
\end{align}

Equating $D'(\alpha) = 0$,

\begin{align}
0 &= 2 - \dfrac{6\cos \alpha}{k\sqrt{1-\dfrac{\sin^2 \alpha}{k^2}}}\nonumber\\
\dfrac{6\cos \alpha}{k\sqrt{1-\dfrac{\sin^2 \alpha}{k^2}}} &= 2
\end{align}

As mentioned in Section \ref{sec:primary}, since $k>1$, then $1-\dfrac{\sin^2 \alpha}{k} >0$ for all $\alpha$.

\begin{align}
\dfrac{6\cos \alpha}{k\sqrt{1-\dfrac{\sin^2 \alpha}{k^2}}} &= 2\nonumber\\
6\cos \alpha &= 2k\sqrt{1-\dfrac{\sin^2 \alpha}{k^2}}\nonumber\\
3\cos \alpha &= k\sqrt{1-\dfrac{\sin^2 \alpha}{k^2}}
\end{align}

Squaring both sides,

\begin{align}
9\cos^2 \alpha &= k^2\left( 1- \dfrac{\sin^2\alpha}{k^2}\right) \nonumber\\
9\cos^2 \alpha &= k^2 - \sin^2 \alpha \nonumber\\
k^2 &= \sin^2 \alpha +9 \cos^2 \alpha \nonumber\\
k^2 &= \sin^2 \alpha + \cos^2 \alpha + 8\cos^2 \alpha
\end{align}

Note that $\sin^2 \alpha + \cos^2 \alpha = 1$.

\begin{align}
k^2 &= 1 + 8\cos^2 \alpha \nonumber\\
\cos^2 \alpha &= \frac{k^2-1}{8} \nonumber\\
\cos \alpha &= \sqrt{\frac{k^2-1}{8}} \nonumber
\end{align}

This shows that $D(\alpha)$ is minimum when 

\begin{equation} \label{eq:cossec}
\cos \alpha = \sqrt{\frac{k^2-1}{8}}
\end{equation}

Substituting $k = 4/3$,

\begin{align*}
\cos \alpha &= \sqrt{\frac{(4/3)^2-1}{8}}\\
&= \sqrt{\frac{16/9-1}{8}}\\
&= \sqrt{\frac{7/9}{8}}\\
\cos \alpha &= \sqrt{\frac{7}{72}}\\
\alpha &= \cos^{-1} \sqrt{\frac{7}{72}}\\
\alpha &\approx 1.2537\\
\alpha &\approx 71.8 \degree\\
\end{align*}

When $k=4/3$, $D(\alpha)$ is minimum when $\alpha = 71.8 \degree$.

\begin{align*}
\min(D(\alpha)) \approx D(1.2537) &= 2\pi + 2(1.2537) - 6 \sin^{-1} \left( \frac{\sin 1.2537}{4/3} \right)\\
&\approx 4.031\\
&\approx 231 \degree\\
\end{align*}

The minimum value of $D(\alpha)$ is $231\degree$.\\

As mentioned in Section \ref{sec:primary}, $\Delta D/ \Delta \alpha \approx 0$ implies that light rays converge at that certain angle, meaning that it is the optimum angle to properly see the rainbow \cite[219]{stewart21}.\\

Let $D_2(\alpha)$ be the clockwise deviation angle of the light ray. This means that $D(\alpha) + D_2(\alpha) = 2\pi = 360\degree$.\\

Since $\min(D(\alpha)) = 231\degree$, then $D_2(\alpha) = 360\degree - 231\degree = 129 \degree$.\\

Therefore, the rainbow angle of the secondary rainbow is $180 \degree - 129\degree = 51\degree$. Note that in Section \ref{sec:primarycolor}, the maximum rainbow angle is on the red bow at $\approx 42.4 \degree$. This means that the secondary rainbow is {\itshape always} above the primary rainbow.

\subsection{Colors in a secondary rainbow}

From Section \ref{sec:secondary}, we get the following equations:

\begin{equation}
\begin{cases}
D(\alpha) + D_2(\alpha) = 2\pi\\
\text{rainbow angle} = \pi - D_2(\alpha) 
\end{cases}
\label{eq:system}
\end{equation}

From the system of equations at \eqref{eq:system}, we can infer that

\begin{equation}
\text{rainbow angle} = D(\alpha) - \pi
\end{equation}

As previously mentioned in Equation \eqref{eq:cossec}, $D(\alpha)$ is minimum when

\begin{align}
\cos \alpha = \sqrt{\frac{k^2-1}{8}} \nonumber\\
\alpha = \cos^{-1} \sqrt{\frac{k^2-1}{8}}
\end{align}

Substituting $k \approx 1.3318$ for red light,

\begin{align*}
\alpha &= \cos^{-1} \sqrt{\frac{1.3318^2-1}{8}}\\
&\approx 1.2546
\end{align*}

Substituting $a \approx 1.2546$ and $k \approx 1.3318$ on $D(\alpha)$ from Equation \ref{eq:derivativesec},

\begin{align*}
D(1.2546) &= 2\pi + 2(1.2546) - 6 \sin^{-1}{\frac{\sin 1.2546}{1.3318}}\\
&\approx 4.0243\\
&\approx 230.6 \degree
\end{align*}

The rainbow angle of the red bow is $\approx 230.6\degree - 180\degree = 50.6\degree$.\\

Substituting $k \approx 1.3435$ for the violet bow,

\begin{align*}
\alpha &= \cos^{-1} \sqrt{\frac{1.3435^2-1}{8}}\\
&\approx 1.2480
\end{align*}

Substituting $a \approx 1.2546$ and $k \approx 1.3318$ on $D(\alpha)$,

\begin{align*}
D(1.2480) &= 2\pi + 2(1.2480) - 6 \sin^{-1}{\frac{\sin 1.2480}{1.3435}}\\
&\approx 4.0772\\
&\approx 233.6 \degree
\end{align*}

The rainbow angle of the violet bow is $\approx 233.6\degree - 180\degree = 53.6\degree$.\\

As observed in the calculations, the rainbow angle of the violet bow ($53.6\degree$) is larger than the rainbow angle of the red bow ($50.6\degree$). This implies the that the secondary rainbow appears violet on top and red on the bottom, inverted of what the primary rainbow is at Section \ref{sec:primarycolor}.

\begin{figure}[H]
\centering
\secondarycolor
\caption{Rainbow angles of the corresponding colors. Note that the one above is the secondary rainbow while the one below is the primary rainbow. Notice that the arrangement of colors are inverted.}
\label{fig:seccolor}
\end{figure}

\section{Conclusion}

Using the principles of light rays, laws of refraction and reflection, the rainbow angle deviations and locations were calculated. Applying the concept of Snell’s Law, the minimum angle of deviation of the primary rainbow was shown, which is at 138 degrees. The angle of deviation of the primary rainbow is the amount of clockwise rotation the ray has undergone, and at minimum angle of deviation, the rainbow is at its brightest as the rays are deviated by approximately the same amount. It is elevated at 42 degrees from the observer to the highest point of the rainbow, which is called the rainbow angle. To explain the colors in the bow, the different refraction index of each color was used. Higher refraction index has the smaller rainbow angle there is, illustrating the order of the seven colors that can be seen from a rainbow.\\
 
Above the primary rainbow is the secondary rainbow that is fainter than the first. The minimum angle of deviation is at 231 degrees, but the angle of deviation in the secondary rainbow is in counterclockwise rotation and is equivalent to 129 degrees in a clockwise rotation. The bow is elevated higher than the primary one, which general rainbow angle is at 51 degrees elevation from the observer to the highest point of the bow. In the secondary rainbow, the rainbow angle gets larger as the index of refraction gets bigger. Consequently, the order of the colors in the secondary rainbow is in opposite of the primary rainbow.

\printbibliography

\end{document}