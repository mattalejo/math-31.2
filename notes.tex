% Options for packages loaded elsewhere
\PassOptionsToPackage{unicode}{hyperref}
\PassOptionsToPackage{hyphens}{url}
%
\documentclass[
]{article}
\usepackage{lmodern}
\usepackage{amssymb,amsmath}
\usepackage{ifxetex,ifluatex}
\ifnum 0\ifxetex 1\fi\ifluatex 1\fi=0 % if pdftex
  \usepackage[T1]{fontenc}
  \usepackage[utf8]{inputenc}
  \usepackage{textcomp} % provide euro and other symbols
\else % if luatex or xetex
  \usepackage{unicode-math}
  \defaultfontfeatures{Scale=MatchLowercase}
  \defaultfontfeatures[\rmfamily]{Ligatures=TeX,Scale=1}
\fi
% Use upquote if available, for straight quotes in verbatim environments
\IfFileExists{upquote.sty}{\usepackage{upquote}}{}
\IfFileExists{microtype.sty}{% use microtype if available
  \usepackage[]{microtype}
  \UseMicrotypeSet[protrusion]{basicmath} % disable protrusion for tt fonts
}{}
\makeatletter
\@ifundefined{KOMAClassName}{% if non-KOMA class
  \IfFileExists{parskip.sty}{%
    \usepackage{parskip}
  }{% else
    \setlength{\parindent}{0pt}
    \setlength{\parskip}{6pt plus 2pt minus 1pt}}
}{% if KOMA class
  \KOMAoptions{parskip=half}}
\makeatother
\usepackage{xcolor}
\IfFileExists{xurl.sty}{\usepackage{xurl}}{} % add URL line breaks if available
\IfFileExists{bookmark.sty}{\usepackage{bookmark}}{\usepackage{hyperref}}
\hypersetup{
  pdftitle={Mathematical Analysis IB},
  hidelinks,
  pdfcreator={LaTeX via pandoc}}
\urlstyle{same} % disable monospaced font for URLs
\usepackage[margin=1in]{geometry}
\usepackage{graphicx}
\makeatletter
\def\maxwidth{\ifdim\Gin@nat@width>\linewidth\linewidth\else\Gin@nat@width\fi}
\def\maxheight{\ifdim\Gin@nat@height>\textheight\textheight\else\Gin@nat@height\fi}
\makeatother
% Scale images if necessary, so that they will not overflow the page
% margins by default, and it is still possible to overwrite the defaults
% using explicit options in \includegraphics[width, height, ...]{}
\setkeys{Gin}{width=\maxwidth,height=\maxheight,keepaspectratio}
% Set default figure placement to htbp
\makeatletter
\def\fps@figure{htbp}
\makeatother
\setlength{\emergencystretch}{3em} % prevent overfull lines
\providecommand{\tightlist}{%
  \setlength{\itemsep}{0pt}\setlength{\parskip}{0pt}}
\setcounter{secnumdepth}{-\maxdimen} % remove section numbering
\ifluatex
  \usepackage{selnolig}  % disable illegal ligatures
\fi

\title{Mathematical Analysis IB}
\author{}
\date{\vspace{-2.5em}}

\begin{document}
\maketitle

\begin{quote}
\href{../math-31.2/notes.pdf}{Download the PDF copy of the notes here}
\end{quote}

\hypertarget{review-on-differentiation}{%
\section*{0 - Review on
differentiation}\label{review-on-differentiation}}
\addcontentsline{toc}{section}{0 - Review on differentiation}

\hypertarget{differentiability}{%
\subsection{Differentiability}\label{differentiability}}

Let \(f\) be a function on some open interval \(I\) containing \(x\).
The derivative of \(f\) at \(x\), denoted by \(f'(x)\), is

\[ f'(x) = \lim_{h\to 0}\frac{f(x+h)-f(x)}{h} \]

\hypertarget{differentiation-rules}{%
\subsection{Differentiation rules}\label{differentiation-rules}}

\begin{enumerate}
\def\labelenumi{\arabic{enumi}.}
\item
  \(\frac{d}{dx}(cf(x))= cf'(x)\)
\item
  \(\frac{d}{dx}(f(x) \pm g(x)) = f'(x) \pm g'(x)\)
\item
  \(\frac{d}{dx}(f(x)g(x)) = f(x)g'(x) + g(x)f'(x)\)
\item
  \(\frac{d}{dx}(\frac{f(x)}{g(x)}) = \frac{g(x)f'(x)-f(x)g'(x)}{(g(x))^2}\)
\item
  \(\frac{d}{dx}(f(g(x))) = f'(g(x))g'(x)\)
\end{enumerate}

\hypertarget{differentiation-formulas-i}{%
\subsection{Differentiation formulas
I}\label{differentiation-formulas-i}}

\begin{enumerate}
\def\labelenumi{\arabic{enumi}.}
\item
  \(\frac{d}{dx}(c)=0, c \in \mathbb{R}\)
\item
  \(\frac{d}{dx}(x^r)=rx^{r-1}, r \in \mathbb{R}\)
\item
  \(\frac{d}{dx}(\sin x)=\cos x\)
\item
  \(\frac{d}{dx}(\cos x)=\sin x\)
\item
  \(\frac{d}{dx}(\tan x)=\sec^2x\)
\item
  \(\frac{d}{dx}(\cot x)=-\csc^2x\)
\item
  \(\frac{d}{dx}(\sec x) = \sec x\tan x\)
\item
  \(\frac{d}{dx}(\csc x)=-\csc x\cot x\)
\end{enumerate}

\hypertarget{differentiation-formulas-ii}{%
\subsection{Differentiation formulas
II}\label{differentiation-formulas-ii}}

\begin{enumerate}
\def\labelenumi{\arabic{enumi}.}
\item
  \(\frac{d}{dx}(e^x) = e^x\)
\item
  \(\frac{d}{dx}(\ln|x|) = \frac{1}{x}\)
\item
  \(\frac{d}{dx}(\sin^{-1}x) = \frac{1}{\sqrt{1-x^2}}\)
\item
  \(\frac{d}{dx}(\tan^{-1}x) = \frac{1}{1+x^2}\)
\item
  \(\frac{d}{dx}(\sec^{-1}x) = \frac{1}{x \sqrt{x^2-1}}\)
\end{enumerate}

\hypertarget{mean-value-theorem}{%
\subsection{Mean value theorem}\label{mean-value-theorem}}

Let \(f\) be a function that is continuous on \([a,b]\) and is
differentiable on \((a,b)\). Then there is a number \(c\in(a,b)\) such
that

\[ f'(c)=\frac{f(b)-f(a)}{b-a} \]

\hypertarget{consequences-of-mvt}{%
\subsection{Consequences of MVT}\label{consequences-of-mvt}}

\hypertarget{zero-derivative}{%
\subsubsection{Zero derivative}\label{zero-derivative}}

If \(f'(x)=0 \;\forall x\) in interval \(I\), then
\(f(x)=c \;\forall x\in I\) for some constant \(C\).

\hypertarget{equal-derivatives}{%
\subsubsection{Equal derivatives}\label{equal-derivatives}}

If \(f'(x)-g'(x)=0 \;\forall x\) in an interval \(I\), then
\(f(x)=g(x)+C\) for some constant \(C\).

\hypertarget{example}{%
\paragraph{Example}\label{example}}

\begin{quote}
Let \(f(x)=\cos^{-1}x\) and \(g(x)=-\sin^{-1}x\)

This implies that \(x \in [-1,1]\) and
\(f(x),g(x) \in [-\frac{\pi}{2},\frac{\pi}{2}]\)

\[ f'(x)=-\frac{1}{\sqrt{x^2+1}}\]

\[ g'(x)=-\frac{1}{\sqrt{x^2+1}} \] Since \(f'(x)-g'(x)=0\) for
\(x \in [-1,1]\), then \(f(x)-g(x)=C\) for some constant \(C\) by a
corollary.

\begin{align*}
\cos^{-1}x - (-\sin^{-1}x)&=C \\
\cos^{-1}x +\sin^{-1}x&=C
\end{align*}

Substituting \(x \in [-1,1]\), in this case, let's use \(x=0\),

\begin{align*}
\cos^{-1}(0) +\sin^{-1}(0)&=C \\
0 + \frac{\pi}{2} &= C \\
C &= \frac{\pi}{2} 
\end{align*}

\[ \therefore \forall x \in[-1,1],f(x)-g(x)= \frac{\pi}{2}\]
\end{quote}

\hypertarget{differentials}{%
\subsection{Differentials}\label{differentials}}

\begin{align*}
f'(x)&=\frac{dy}{dx}\\
f'(x)dx &= dy
\end{align*}

\hypertarget{indefinite-and-definite-integrals}{%
\section{1 - Indefinite and definite
integrals}\label{indefinite-and-definite-integrals}}

\hypertarget{indefinite-integral}{%
\subsection{Indefinite integral}\label{indefinite-integral}}

The main interpretation of derivative is the slope of a tangent line of
a curve.

\hypertarget{example-1}{%
\paragraph{Example}\label{example-1}}

\begin{quote}
At any point \((x,y)\) on a particular curve \(y=F(x)\), the tangent
line has a slope equal to \(4x-5\). If the curve contains the point
\((3,7)\), find \(F(x)\).

\textbf{Solution.} Since the slope is equal to \(4x-5\) for any point
\((x,y)\), then the slope at \((3,7)\) is \(4(3)-5=7\).

\(4x-5\) therefore represents the tangent slope for all values of \(x\).
So

\[ F'(x)=4x-5 \]

By intuition, we can conclude that \(F(x)=2x^2-5x\).

However given \(F(x)=2x^2-5x+1\), \(F'(x)\) remains the same. And so is
\(F(x)=2x^2-5x-3\), \(F(x)=2x^2-5x+\pi\), and infinitely more functions.
We can arbitrarily assign a constant \(k\), so that \(F(x)=2x^2-5x+k\).

Substituting \((x,y)=(3,7),\)

\begin{align*}
7 &= 2(3)^2 - 5(3) +k\\
7 &= 18 - 15 + k \\
k &= 4 
\end{align*}

So \(F(x)=2x^2-5x+4\).
\end{quote}

\hypertarget{definition-of-an-antiderivative}{%
\subsubsection{Definition of an
antiderivative}\label{definition-of-an-antiderivative}}

A function \(F\) is called an antiderivative of the function \(f\) on an
interval \(I\) if \(F'(x) = f(x) \;\forall x \in I\).

\(F(x)=2x^2-5x\) is a \textbf{possible} antiderivative of \(f(x)=4x-5\).
\(F(x)=2x^2-5x+4\) is also a \textbf{possible} antiderivative of
\(f(x)=4x-5\).

\hypertarget{equal-derivatives-1}{%
\subsubsection{Equal derivatives}\label{equal-derivatives-1}}

If \(F'(x)=G'(x) \;\forall x\) in an interval \(I\), then
\(F(x) = G(x) + C \;\forall x \in I\) for some constant \(C\).

\hypertarget{integration-notation}{%
\subsubsection{Integration notation}\label{integration-notation}}

The collection of all antiderivatives of \(f\) is denoted by

\[ \int f(x)dx \]

which is read as ``the integral of \(f(x)dx\).''

This collection is also called the \textbf{indefinite integral} of
\(f\).

The reverse process if differentiation is called
\textbf{antidifferentiation} or \textbf{integration}.

\(\int (4x-5)dx = 2x^2-5x+C\) for some constant \(C\).

\(C\) is the constant of integration.

\(\int \sin xdx = -\cos x+C\)

\hypertarget{integration-rules}{%
\subsubsection{Integration rules}\label{integration-rules}}

\begin{enumerate}
\def\labelenumi{\arabic{enumi}.}
\item
  \(\int kf(x)dx = k \int f(x)dx\), \(k\) constant
\item
  \(\int f(x) \pm g(x) dx = \int f(x)dx \pm \int g(x)dx\)
\end{enumerate}

\hypertarget{integration-formulas-i}{%
\subsubsection{Integration formulas I}\label{integration-formulas-i}}

\begin{enumerate}
\def\labelenumi{\arabic{enumi}.}
\item
  \(\int kdx = kx + C, k \in \mathbb{R}\)
\item
  \(\int x^ndx = \frac{x^{n+1}}{n+1}+C, n \in \mathbb{R}, n \neq -1\)
\end{enumerate}

\hypertarget{integration-formulas-ii}{%
\subsubsection{Integration formulas II}\label{integration-formulas-ii}}

\begin{enumerate}
\def\labelenumi{\arabic{enumi}.}
\item
  \(\int \sin xdx = -\cos x + C\)
\item
  \(\int \cos xdx = \sin x + C\)
\item
  \(\int \sec^2xdx = \tan x + C\)
\item
  \(\int \csc^2xdx = -\cot x + C\)
\item
  \(\int \sec x\tan xdx = \sec x + C\)
\item
  \(\int \csc x\cot xdx = -\csc x +C\)
\end{enumerate}

\hypertarget{integration-formulas-iii}{%
\subsubsection{Integration formulas
III}\label{integration-formulas-iii}}

\begin{enumerate}
\def\labelenumi{\arabic{enumi}.}
\item
  \(\int e^xdx =e^x +C\)
\item
  \(\int \frac{1}{x}dx=\ln|x|+C\)
\item
  \(\int \frac{1}{\sqrt{1-x^2}}dx = \sin^{-1}x +C\)
\item
  \(\int \frac{1}{1+x^2}dx = \tan^{-1}x +C\)
\item
  \(\int \frac{1}{x\sqrt{x^2-1}}dx= \sec^{-1}+C\)
\end{enumerate}

\hypertarget{substitution-rule}{%
\subsection{Substitution rule}\label{substitution-rule}}

\hypertarget{chain-rule-for-derivatives}{%
\subsubsection{Chain rule for
derivatives}\label{chain-rule-for-derivatives}}

\[ \frac{d}{dx}(f(g(x)))=f'(g(x))g'(x) \]

If follows that

\[ \int f'(g(x))g'(x)dx = f(g(x))+C \]

\hypertarget{example-2}{%
\paragraph{Example}\label{example-2}}

\begin{quote}
Evaluate \(\int 2xcosx^2dx\).

\textbf{Preliminary work.} By intuition, we can get \(f(x)=sinx\) and
\(g(x)=x^2\)

\[ \int 2x\cos x^2dx = f(g(x)) = \sin x^2 \]

\textbf{Solution.} Suppose that \(f'(x)=\frac{dy}{dx}\)

\[ dy = f'(x)dx \]

Let \(u=g(x)\), then \(g'(x)=\frac{du}{dx}\)

\[ du = g'(x)dx \]

Let \(u=x^2\)

\begin{align*}
du &=2xdx\\
 \int 2x\cos x^2dx &= \int \cos udu\\
 &=\sin u+C \\
 &=\sin x^2+C
\end{align*}
\end{quote}

\hypertarget{definition-of-the-substitution-rule}{%
\subsubsection{Definition of the substitution
rule}\label{definition-of-the-substitution-rule}}

If \(u=g(x)\) is a differentiable function whose range is interval \(I\)
and \(f\) is continuous on \(I\), then

\[ \int f'(g(x))g'(x) = \int f(u)du \]

\hypertarget{the-fundamental-theorem-of-calculus}{%
\subsection{The Fundamental Theorem of
Calculus}\label{the-fundamental-theorem-of-calculus}}

\hypertarget{the-area-problem}{%
\subsubsection{The area problem}\label{the-area-problem}}

Let \(f\) be a continuous nonnegative function on \([a,b]\). Find the
area of the regiom bounded by the curve \(y=f(x)\), the lines \(x=a\),
\(x=b\), and the \(x\)-axis.

\begin{quote}
The area is often coined the \textbf{region under the curve}, which
generally means the area in between the curve and the \(x\)-axis
\end{quote}

\hypertarget{example-3}{%
\paragraph{Example}\label{example-3}}

\begin{quote}
Consider \(f(x) = x^2 +1\) on \([0,2]\).

\textbf{Solution.} Let \(A\) be the area under the curve

Using right endpoints (5 rectangles)

\[ \Delta x = \frac{2-0}{5} = \frac{2}{5} = 0.4 \] Rectangle 1:
\((\Delta x)(f(0.4)) = (0.4)(1.16)\)

Rectangle 2: \((\Delta x)(f(0.8)) = (0.4)(1.64)\)

Rectangle 3: \((\Delta x)(f(1.2)) = (0.4)(2.44)\)

Rectangle 4: \((\Delta x)(f(1.6)) = (0.4)(3.56)\)

Rectangle 5: \((\Delta x)(f(2.0)) = (0.4)(5)\)

\[ A_5^+ = (0.4)(1.16) + (0.4)(1.64) + (0.4)(2.44) + (0.4)(3.56) +(0.4)(5) = 5.52 \]
\(A_5^+\) is an overestimation of \(A\).

Using left endpoints (5 rectangles):

Rectangle 1: \((\Delta x)(f(0)) = (0.4)(1)\)

Rectangle 2: \((\Delta x)(f(0.4)) = (0.4)(1.16)\)

Rectangle 3: \((\Delta x)(f(0.8)) = (0.4)(1.64)\)

Rectangle 4: \((\Delta x)(f(1.2)) = (0.4)(2.44)\)

Rectangle 5: \((\Delta x)(f(1.6)) = (0.4)(3.56)\)

\[ A_5^- = (0.4)(1) + (0.4)(1.16) + (0.4)(1.64) + (0.4)(2.44) + (0.4)(3.56) + =3.92 \]

\(A_5^-\) is an underestimation of \(A\)

We can increase the number of rectangles and compute the area \(A\) more
\textbf{accurately} by computing the area as the number of rectangles
approach infinity.

Let the number of rectangles be \(n\)

\[ \Delta x = \frac{2-0}{n}=\frac{2}{n} \] Let \(x_0\) be the first
point: \(x_i=o\)

\begin{align*}
x_1 &= \frac{2}{n} & x_2 &= \frac{4}{n} & x_3 &= \frac{6}{n}\\
x_4 &= \frac{8}{n} & x_5 &= \frac{10}{n} & x_6 &= \frac{12}{n}\\
x_7 &= \frac{14}{n} &    &\cdots      & x_i &= \frac{2i}{n}
\end{align*}

\begin{align*}
A_n^+ &= R_1 + R_2 + R_3 + R_4 + \cdots + R_n\\
&= \sum_{i=0}^n \Delta x (f(x_i))\\
&= \sum_{i=0}^n \frac{2}{n} ((\frac{2i}{n})^2 +1)\\
&= \frac{2}{n} \sum_{i=0}^n (\frac{4i^2}{n^2} +1)\\
&= \frac{2}{n} \Big[\sum_{i=0}^n (\frac{4i^2}{n^2}) + \sum_{i=0}^n(1)\Big]\\
&= \frac{2}{n} \Big[\frac{4}{n^2}\sum_{i=0}^n (i^2) + \sum_{i=0}^n(1)\Big]\\
&= \frac{2}{n} \Big[\frac{4}{n^2} \Big(\frac{(n)(n+1)(2n+1)}{6}\Big)+n\Big]\\
A_n&= \frac{8}{n^3} \frac{(n)(n+1)(2n+1)}{6} +2

\end{align*}

\begin{align*}
A &= \lim_{n \to\infty} A_n\\
&= \lim_{n \to \infty}  \frac{8}{n^3} \frac{(n)(n+1)(2n+1)}{6} + 2\\
&= \lim_{n \to \infty} \frac{8(n)(n+1)(2n+1)}{6n^3} +2\\
&= \lim_{n \to \infty} \frac{8(n+1)(2n+1)}{6n^2} +2\\
&= \frac{16}{6} +2\\
&= \frac{28}{6} = \frac{14}{3}
\end{align*}
\end{quote}

Let \(x_0=a, x_1,x_2, \ldots, x_n = b\),

\[A_n = \sum_{i=1}{n}f(x_i^*)\Delta x \]

This is also called the \textbf{Riemann sum}.

\hypertarget{definite-integral-and-integrability}{%
\subsubsection{Definite integral and
integrability}\label{definite-integral-and-integrability}}

The definite integral of \(f\) from \(a\) to \(b\) is

\[ \int_a^b f(x)dx = \lim_{x\to\infty}\sum_{i=1}{n}f(x_i^*)\Delta x \]
provided that such limit exists.

We say that \(f\) is integrable on \([a,b]\)

\hypertarget{remarks}{%
\paragraph{Remarks}\label{remarks}}

\begin{enumerate}
\def\labelenumi{\arabic{enumi}.}
\item
  If a function is continuous on \([a,b]\), it is integrable on
  \([a,b]\).
\item
  If \(f\) is a nonnegative continuous function on \([a,b]\), then
  \(\int_a^b f(x)dx\) is the area under the curve \(y=f(x)\) from
  \(x=a\) and \(x=b\)
\item
\end{enumerate}

\hypertarget{conventions-on-definite-integral}{%
\subsubsection{Conventions on definite
integral}\label{conventions-on-definite-integral}}

\begin{enumerate}
\def\labelenumi{\arabic{enumi}.}
\item
  \(\int_b^a f(x)dx = -\int_a^b f(x)dx\)
\item
  \(\int_a^a f(x)dx = 0\)
\end{enumerate}

\hypertarget{properties-of-the-definite-integral}{%
\subsubsection{Properties of the definite
integral}\label{properties-of-the-definite-integral}}

\begin{enumerate}
\def\labelenumi{\arabic{enumi}.}
\item
  \(\int_a^b cf(x)dx = c\int_a^b f(x)dx\)
\item
  \(\int_a^b [f(x) \pm g(x)] dx = \int_a^b f(x) \pm \int_a^b g(x)\)
\item
\item
\item
\item
\end{enumerate}

\hypertarget{proof-of-the-fundamental-theorem-of-calculus}{%
\subsection{Proof of the Fundamental Theorem of
Calculus}\label{proof-of-the-fundamental-theorem-of-calculus}}

\hypertarget{the-area-problem-1}{%
\subsection{The area problem}\label{the-area-problem-1}}

\hypertarget{the-definite-integrals}{%
\subsection{The definite Integrals}\label{the-definite-integrals}}

\hypertarget{application-i}{%
\section{2 - Application I}\label{application-i}}

\hypertarget{areas-between-curves}{%
\subsection{Areas between curves}\label{areas-between-curves}}

\hypertarget{volumes-and-volumes-of-revolution-using-disks-and-washers}{%
\subsection{Volumes and volumes of revolution using disks and
washers}\label{volumes-and-volumes-of-revolution-using-disks-and-washers}}

\hypertarget{volumes-of-solids-of-revolution-using-cylindrical-shells}{%
\subsection{Volumes of solids of revolution using cylindrical
shells}\label{volumes-of-solids-of-revolution-using-cylindrical-shells}}

\hypertarget{techniques-of-integration}{%
\section{3 - Techniques of
integration}\label{techniques-of-integration}}

\hypertarget{integration-by-parts}{%
\subsection{Integration by parts}\label{integration-by-parts}}

\hypertarget{trigonometric-integrals}{%
\subsection{Trigonometric integrals}\label{trigonometric-integrals}}

\hypertarget{trigonometric-substitution}{%
\subsection{Trigonometric
substitution}\label{trigonometric-substitution}}

\hypertarget{partial-fractions}{%
\subsection{Partial fractions}\label{partial-fractions}}

\hypertarget{applications-ii}{%
\section{4 - Applications II}\label{applications-ii}}

\hypertarget{arc-length}{%
\subsection{Arc length}\label{arc-length}}

\hypertarget{variable-separable-differential-equations-and-models-for-population-growth}{%
\subsection{Variable-separable differential equations and models for
population
growth}\label{variable-separable-differential-equations-and-models-for-population-growth}}

\end{document}
